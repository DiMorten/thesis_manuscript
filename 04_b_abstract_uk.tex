\abstractuk{
Official deforestation monitoring in the Brazilian Amazon is currently done by human experts who visually evaluate remote sensing images and label each individual pixel as deforestation or no deforestation. That methodology is obviously costly and time-consuming. The reason for not using fully automatic methods for the task is the need for the highest possible accuracies in the authoritative deforestation figures. In this work, a semi-automatic, deep learning-based alternative is proposed, in which a deep neural network is first trained with existing images and references from previous years, and employed to perform deforestation detection on recent images. After inference, the uncertainty in the network’s pixel-level results is estimated, and it is assumed that low-uncertainty classification results can be trusted. The remaining high-uncertainty regions, which correspond to a small percentage of the test area, are then submitted to post classification, e.g., an auditing procedure carried out visually by a human specialist. In this way, the manual labeling effort is greatly reduced. The evaluation of such methodology for secondary vegetation mapping in the Brazilian Amazon and Cerrado biomes is also planned. Different uncertainty methods will be compared, including confidence-based approaches, Monte Carlo Dropout (MCD), ensembles and evidential learning. Convolutional and transformer based networks will be used as base models. In this document, preliminary results are presented for deforestation detection. Assuming that the visual evaluation is correct, the accuracies obtained thus far with such a methodology for deforestation detection in two challenging study areas in the Brazilian Amazon are very high -- of up to 96.9\% average $F_1$-score, for an area to be audited of only 3\%. The current code is available at \url{https://github.com/DiMorten/deforestation_uncertainty}.
}

