%% -*- coding: utf-8; -*-

\documentclass[phd, american]{ThesisPUC}

\usepackage{tabularx}
\usepackage{multirow}
\usepackage{multicol}
\usepackage{colortbl}
\usepackage[dvipsnames, svgnames, x11names, fixpdftex, table,xcdraw]{xcolor}
\usepackage{numprint}
\usepackage{textcomp}
\usepackage{booktabs}
\usepackage{amsmath}
\usepackage{enumitem}
\usepackage{amssymb}
\usepackage{textcomp}
\usepackage{graphicx,subfigure}
\usepackage{float}
\usepackage{footnote}
\usepackage[square,numbers,sort&compress]{natbib}

\makeatletter
\renewcommand\@biblabel[1]{#1 }
\makeatother

%% DOCUMENT FORMAT ============================================================================
\newcommand{\Rset}{\mathbb{R}}
\newcommand{\Zset}{\mathbb{Z}}
\newcommand{\tabitem}{~~\llap{\textbullet}~~}

\newcolumntype{L}[1]{>{\raggedright\let\newline\\\arraybackslash\hspace{0pt}}m{#1}}

\npthousandsep{}
\npdecimalsign{,}

\tablesmode{figtab} %% [nada, fig, tab ou figtab]
\setcounter{tocdepth}{3}
\setcounter{secnumdepth}{3}

\newlength\mylen
\cftsetindents{figure}{0em}{0em}
\cftsetindents{table}{0em}{0em}

\usepackage{chngcntr}
\counterwithout{figure}{chapter}
\counterwithout{table}{chapter}

%% DOCUMENT INFORMATION =======================================================================
\title{Incerteza no Mapeamento de Desmatamento e Regeneração Baseado em Aprendizado Profundo}
\titleuk{Uncertainty in Deep Learning-Based Deforestation and Regeneration Mapping}

\author{Jorge Andres Chamorro Martinez}
\authorR{Chamorro J., A.}

\advisor{Raul Queiroz Feitosa}{Prof. Dr.}
\advisorR{Feitosa, R. Q.}

%\coadvisor{Ieda Del'Arco Sanches}{Dra.}
%\coadvisorInst{National Institute for Space Research}{INPE}
%\coadvisorR{Sanches, I. D.}

\jury{
    \jurymember{Gilberto Camara}{Prof. Dr.}{Instituto Nacional de Pesquisas Espaciais}{INPE}
    \jurymember{Wesley Nunes Goncalves}{Prof. Dr.}{Universidade Federal de Mato Grosso do Sul}{UFMGS}
    \jurymember{Gilson Alexandre Ostwald Pedro da Costa}{Prof. Dr.}{Universidade do Estado do Rio de Janeiro}{UERJ}    
%    \jurymember{Ruy Luiz Milidiú}{Prof.}{Pontifical Catholic University of Rio de Janeiro}{PUC-Rio}
%    \jurymember{Rodrigo C. de Lamare}{Prof.}{Pontifical Catholic University of Rio de Janeiro}{PUC-Rio}
%    \schoolhead{Márcio da Silveira Carvalho}{Prof.}
}

\university{Pontifícia Universidade Católica do Rio de Janeiro}
\uni{PUC-Rio}
\school{Centro Técnico Científico}
\department{Engenharia Elétrica}
\program{Engenharia Elétrica}

\city{Rio de Janeiro}
\day{28}
\month{February}
\year{2023}

%\abreviationstitleuk {List of Symbols and Abbreviations}

% \dedication {To all those whose have a dream and ventured to make it true.}

%% DOCUMENT STRUCTURE =========================================================================
%-- Author's resume...
\resume{
  The author received his bachelor's degree in Electronic Engineering at the University of Nariño in 2015. He obtained his master's degree in Electrical Engineering with emphasis on Signal Processing and Control at the Pontifícia Universidade Católica do Rio de Janeiro (PUC-Rio) in 2019. Since then, he has worked in the fields of Machine Learning, Digital Image Processing and Remote Sensing.}

  

%-- Catalog prekeys...
\CDD{621.3}
\catalogprekeywords{\catalogprekey{Engenharia Elétrica}}
\usecolour{true}

%-- Acknowledgment...
\acknowledgment{


I would like to thank my advisor, Prof. Raul Queiroz Feitosa, for all the help, support, and friendship. I would also like to thank all the members of the Computer Vision Lab (LVC) research group for their help and their friendship. I acknowledge the financial support from CNPq.

  }

%-- Abstract...
\abstract{
%Atualmente, o monitoramento oficial do desmatamento na Amazônia brasileira é feito por especialistas humanos que avaliam visualmente imagens de sensoriamento remoto e rotulam cada pixel individual como desmatamento ou não desmatamento. Essa metodologia é obviamente cara e demorada. A razão para não usar métodos totalmente automáticos para a tarefa é a necessidade da maior precisão possível nos números oficiais de desmatamento. Neste trabalho, propomos uma alternativa semiautomática baseada em aprendizado profundo, na qual primeiro treinamos uma rede totalmente convolucional com imagens e referências existentes de anos anteriores e empregamos essa rede para realizar a detecção de desmatamento em imagens recentes. Após a inferência, estimamos a incerteza nos resultados em nível de pixel da rede e assumimos que os resultados da classificação de baixa incerteza podem ser confiáveis. As demais regiões de alta incerteza, que correspondem a uma pequena porcentagem da área de teste, são então submetidas a um procedimento de auditoria, realizado visualmente por um especialista humano. Desta forma, o esforço de rotulagem manual é bastante reduzido. Também avaliamos a metodologia de mapeamento da vegetação secundária nos biomas Amazônia e Cerrado. Diferentes métodos de incerteza foram comparados, incluindo abordagens baseadas em confiança, Monte Carlo Dropout (MCD), ensembles e deep learning evidencial. Redes convolucionais e baseadas em transformadores foram usadas como modelos de base. Resultados preliminares são apresentados para detecção de desmatamento. Supondo que a avaliação visual esteja correta, as acurácias obtidas com tal metodologia nas áreas de teste são muito altas -- de até 96,9\% média $F_1$-score, para uma área a ser auditada de apenas 3\%. Avaliamos a metodologia proposta em duas áreas de estudo desafiadoras na Amazônia brasileira. O código está disponível em \url{https://github.com/DiMorten/deforestation_uncertainty}.

Atualmente, o monitoramento oficial do desmatamento na Amazônia brasileira é feito por especialistas humanos que avaliam visualmente imagens de sensoriamento remoto e rotulam cada pixel individual como desmatamento ou não desmatamento. Essa metodologia é obviamente cara e demorada. A razão para não usar métodos totalmente automáticos para a tarefa é a necessidade da maior precisão possível nos números oficiais de desmatamento. Neste trabalho, é proposta uma alternativa semiautomática baseada em aprendizado profundo, na qual uma rede neural profunda é primeiro treinada com imagens e referências existentes de anos anteriores e empregada para realizar a detecção de desmatamento em imagens recentes. Após a inferência, a incerteza nas predições da rede é estimada e assume-se que os resultados com baixa incerteza são confiáveis. As regiões restantes de alta incerteza, que correspondem a uma pequena porcentagem da área de teste, são então submetidas à pós-classificação, por exemplo, em um procedimento de auditoria realizado visualmente por um especialista humano. Desta forma, o esforço de rotulagem manual é bastante reduzido. A avaliação dessa metodologia para o mapeamento da vegetação secundária nos biomas Amazônia e Cerrado também está prevista. Diferentes métodos de incerteza serão comparados, incluindo abordagens baseadas em confiança, Monte Carlo Dropout (MCD), \textit{ensembles} e \textit{evidencial learning}. Redes totalmente convolucionais e \textit{transformers} serão usadas como arquiteturas básicas. Neste documento são apresentados resultados preliminares para detecção de desmatamento. Supondo que a avaliação visual esteja correta, as acurácias obtidas até agora com essa metodologia para detecção de desmatamento em duas áreas de estudo desafiadoras na Amazônia brasileira são muito altas -- de até 96,9\% média $F_1$-score, para uma área a auditar de apenas 3\%. O código atual está disponível em \url{https://github.com/DiMorten/deforestation_uncertainty}.
}
\abstractuk{
Official deforestation monitoring in the Brazilian Amazon is currently done by human experts who visually evaluate remote sensing images and label each individual pixel as deforestation or no deforestation. That methodology is obviously costly and time-consuming. The reason for not using fully automatic methods for the task is the need for the highest possible accuracies in the authoritative deforestation figures. In this work, a semi-automatic, deep learning-based alternative is proposed, in which a deep neural network is first trained with existing images and references from previous years, and employed to perform deforestation detection on recent images. After inference, the uncertainty in the network’s pixel-level results is estimated, and it is assumed that low-uncertainty classification results can be trusted. The remaining high-uncertainty regions, which correspond to a small percentage of the test area, are then submitted to post classification, e.g., an auditing procedure carried out visually by a human specialist. In this way, the manual labeling effort is greatly reduced. The evaluation of such methodology for secondary vegetation mapping in the Brazilian Amazon and Cerrado biomes is also planned. Different uncertainty methods will be compared, including confidence-based approaches, Monte Carlo Dropout (MCD), ensembles and evidential learning. Convolutional and transformer based networks will be used as base models. In this document, preliminary results are presented for deforestation detection. Assuming that the visual evaluation is correct, the accuracies obtained thus far with such a methodology for deforestation detection in two challenging study areas in the Brazilian Amazon are very high -- of up to 96.9\% average $F_1$-score, for an area to be audited of only 3\%. The current code is available at \url{https://github.com/DiMorten/deforestation_uncertainty}.
}



%-- Keywords...
\keywords{
    \key{Reconhecimento de Culturas;}
    \key{Sensoriamento Remoto;}
    \key{Modelos Graficos Probabilisticos;}
    \key{Imagens óticas;}
    \key{Imagens de radar}
}
\keywordsuk{
    \key{Crop Recognition;}
    \key{Remote Sensing;}
    \key{Probabilistic Graphical Models;}
    \key{Optical imagery;}
    \key{Radar imagery}
}

% \epigraph {All men have stars, but they are not the same things for different people. For some, who are travelers, the stars are guides. For others they are no more than little lights in the sky. For others, who are scholars, they are problems... But all these stars are silent. You—you alone will have stars as no one else has them.}
% \epigraphauthor {Antoine de Saint-Exupéry}
% \epigraphbook {The Little Prince} 

%-- Textual elements...
\begin{document}
   %% -*- coding: utf-8; -*-

\begin{thenotations}
	\noindent	
	\begin{tabular}{ll}	
        PRODES & Amazon Deforestation Monitoring Project \\
        INPE & National Space Research  Institute \\
        SAR & Synthetic Aperture Radar \\
        MCD & Monte Carlo Dropout \\
        DPT & Dense Prediction Transformer \\
        % DINO & DETR with Improved DeNoising Anchor Boxes for End-to-End Object Detection \\
        Swin & Shifted Windows \\
        
        CNN & Convolutional Neural Network \\
        ROI & Region Of Interest \\
        NLP & Natural Language Processing \\
        MNIST & Modified National Institute of Standards and Technology \\
        ViT & Vision Transformer \\
        FCN & Fully Convolutional Network \\
        MHA & Multi-Head Attention \\
        
        TTA & Test-Time Augmentation \\
        MI & Mutual Information \\
        KL & Kullback-Leibler \\
        ReLU & Rectified Linear Unit \\
        MLP & Multi-Layer Perceptron \\
        DETR & End-to-End Object Detection with Transformers \\
        MT & Mato Grosso \\
        PA & Para \\
        AA & Alert Area \\
        SITS & Satellite Image Time Series \\
	    \iffalse
        AP & Association Potential \\    
	    BN & Batch Normalization \\
	    CM & Confusion Matrix \\
	    CNN & Convolutional Neural Networks \\
	    CRF & Conditional Random Fields \\
	    DEM & Digital Elevation Model \\
	    DL & Deep Learning \\
	    EMBRAPA & Brazilian Agricultural Research Corporation \\
	    EO & Earth Observation \\
	    ESA & European Space Agency \\
	    EVI & Enhanced Vegetation Index \\
	    FCN & Fully Convolutional Networks \\
	    GDP & Gross Domestic Product \\
		GHG & GreenHouse Gas \\
		GLCM & Gray-Level Co-occurrence Matrix \\
		GRD & Ground Range Detected \\
		H & Horizontal \\
		HH & Horizontal-Horizontal \\
		HMM & Hidden Markov Models \\
		HV & Horizontal-Vertical \\
		INPE & National Institute for Space Research \\
		IR & Infrared \\
		IVM & Import Vector Machines \\
		IWS & Interferometric Wide Swath \\
		KNN & K-Nearest Neighbor \\
		LBP & Loopy Belief Propagation \\
		LEN & Luiz Eduardo Magalhães \\
		LiDAR & Light Detection and Ranging \\
		LSTM & Long Short-Term Memory \\
		MRF & Markov Random Fields \\
		MSI & MultiSpectral Instrument \\
        \fi
	\end{tabular}	
	\\
	\begin{tabular}{ll}		
        \iffalse
	    NCC & Non-Commercial Crops \\
		NDVI & Normalized Difference Vegetation Index \\
		NN & Neural Network \\
		OBIA & Object-Based Image Analysis \\
		PGM & Probabilistic Graphical Models \\
		RADAR & RAdio Detection And Ranging \\
		RF & Random Forest \\
		RNN & Recurrent Neural Networks \\
		RS & Remote Sensing \\
		SAR & Synthetic Aperture Radar \\
		Sen2-Agri & Sentinel-2 for Agriculture \\
		SENSAGRI & Sentinels Synergy for Agriculture \\
		SIP & Spatial Interaction Potential \\
		SRTM & Shuttle Radar Topography Mission \\
		SVM & Support Vector Machines \\
		TIP & Temporal Interaction Potential \\
		TWDTW & Time-Weighted Dynamic Time Warping \\
		USGS & United States Geological Survey \\
		UTM & Universal Transverse Mercator \\
		UV & Ultraviolet \\
		V & Vertical \\
		VH & Vertical-Horizontal \\
		VV & Vertical-Vertical \\
        \fi
	\end{tabular}	
\end{thenotations}
   \chapter{INTRODUCTION}

\section{Motivation}
\IEEEPARstart{T}{he} Amazon Deforestation Monitoring Project (PRODES), developed and operated by the Brazilian National Space Research Institute (INPE), provides consistent annual deforestation reports since 1988 \cite{prodes}. In the current PRODES methodology, deforestation is visually inspected and manually annotated over a set of satellite images covering the entire Brazilian Amazon biome every year, which spans an area of 4 million $km^2$. Although costly and time consuming, the current PRODES methodology can be justified by the high accuracies expected for the official, authoritative deforestation figures delivered by such a program \cite{prodes,laurance2002predictors}. Similarly, secondary vegetation mapping refers to mapping the regenerated vegetation areas inside previously deforested zones. The TerraClass project \cite{almeida2016high} provides estimates of secondary vegetation mapping based on a semi-automated approach, where classification results are 100\% audited by experts.


Nevertheless, many studies have focused on automatic deforestation detection from remote sensing imagery. Recent works used fully convolutional deep neural networks for deforestation detection, having as inputs pairs of Sentinel-2 or Landsat-8 multispectral images from consecutive years (e.g. \cite{ortega2021comparison, adarme2022multi, torres2021deforestation}). Although those works deliver considerably high accuracies, in the order of 79.6 to 81.7\% in terms of F1-Score, they do not indicate the uncertainties in their  predictions, which may be over or under confident. Some works have studied automated secondary vegetation mapping. A recent work used SAR data for secondary vegetation mapping in Brazil and Peru \cite{kiyohara2022mapping}. Although their results produced up to 84.2\% overall accuracy, they did not consider uncertainty estimation.


Notwithstanding, multiple uncertainty estimation techniques have been proposed to date \cite{gawlikowski2021survey}. Monte Carlo Dropout (MCD) \cite{gal2016dropout} is an uncertainty estimation technique in which dropout is used at inference time to produce slightly different outcomes at each run. Thereafter, statistical measures are applied to a predefined number of inference runs. That method is particularly efficient because the deep neural network is trained only once. Another possibility is to derive uncertainty values using an ensemble of networks trained with different weight initializations, in that case, the training procedure must be carried out a number of times. A more recent alternative is evidential deep learning \cite{sensoy2018evidential}. Committee based approaches such as MCD and ensembles are computationally expensive due to the need to train or run inference multiple times. Conversely, evidential learning estimates uncertainty using a single training run and a single forward pass. It considers the predictions of a single deterministic network as a subjective opinion, and explicitly models the weight uncertainties. The aforementioned uncertainty estimation techniques generally use convolutional networks as a base models. Nevertheless, recent works have reported high semantic segmentation accuracies using vision transformers as alternatives for convolutional models \cite{ranftl2021vision,zhang2022dino}, but no uncertanty estimation research based on transformers can, however, be found in the literature thus far.
%This research will compare uncertainty estimation techniques obtained with fully convolutional and transformer architectures.


% evidence leading to all possible opinions

% add aleatoric and model uncertainty. talk about domain shift. 
% \cite{gawlikowski2021survey, guo2017calibration}. 
% softmax scores produced at the output do not represent model certainty. Instead, they are normalized model scores without probabilistic meaning \cite{}.


 % in the Brazilian Amazon. 
In short, it is assumed that low-uncertainty predictions are more likely to be incorrect. So an uncertainty map can be used to separate the pixel-level classification outcomes into low-uncertainty and high-uncertainty predictions. The high-uncertainty predictions can then be submitted to a human specialist in an auditing procedure, in which it is expected that the respective regions are classified correctly. The basic idea is to  expressively reduce the need for human intervention in the deforestation and secondary vegetation mapping processes, while achieving very high classification accuracies. Indeed, in preliminary results for deforestation detection, the auditing process was restricted to a mere 3\% of the total area of the study areas, and, assuming that the human specialist is always right, $F_1$ scores considering the whole test areas were of up to 96.9\%.
%, corresponding to the percentage of samples with high uncertainty. Our methodology produces high quality results, with $F_1$ score of up to 95.3 for the low uncertainty samples, and $F_1$ score of up to 96.9 if an expert auditor corrected the high uncertainty samples corresponding to 3\% of the test area. We used a fully convolutional ResUnet as a base network. 

%Furthermore, recent works on deforestation detection trained and tested the networks on the same date. Instead, in this work we simulated a real-world operational application by training the network on dates from the past and making inference on a new upcoming date which was not used by the network during training. We assessed the proposed methodology in two study areas in the states of Para and Mato Grosso in the Brazilian Amazon.

The above-mentioned results are obtained in a real-world operational setup, in which images and references from the past years were used for training the classifier, which was then applied to a pair of recent anniversary images. In the preliminary results for deforestation detection, two sites in the Brazilian states of Pará and Mato Grosso were selected as study areas.

In conclusion, a way to exploit uncertainty estimation in the tasks of deforestation detection and secondary vegetation mapping is proposed in this work. Multiple uncertainty estimation methods will be assessed, including confidence-based approaches, MCD, ensembles and evidential learning. For the dense classificarion models, fully convolutional and transformer architecture models will be considered. Preliminary results are presented for MCD and classifier ensembles as uncertainty estimation techniques, and a ResUnet-based network as the classification model. Preliminary results refer to one of the target applications, namely deforestation detection.


\section{Objectives}
\subsection{General Objective}

The general objective of this work is to propose a methodology that exploits uncertainty measures derived from deep learning-based semantic segmentation models, which can deliver very high accuracies in deforestation and regeneration mapping through a semi-automatic procedure that relies on little human intervention.

\subsection{Specific Objectives}

The specific objectives of this work are the following:
\begin{enumerate}
\item Propose an operational methodology that exploits uncertainty estimation in the problems of deforestation and regeneration mapping from satellite images, aiming at strongly reducing human intervention in the execution of those tasks.
\item Evaluate uncertainty estimation methods such as confidence-based approaches, multi-output approaches and evidential learning, and compare different uncertainty metrics such as predictive entropy, predictive variance and mutual information.
\item Propose and  transformer-based networks such as DPT \cite{ranftl2021vision}, DINO \cite{zhang2022dino} and Swin \cite{liu2021swin} as alternatives for fully convolutional models (e.g, ResUNet) in the context of the proposed methodology.
\item Evaluate the proposed methodology in two applications: deforestation detection in the Brazilian Amazon and secondary vegetation mapping in the Brazilian Amazon and Cerrado biomes.

\end{enumerate}



\section{Contributions}

The main contributions of this work are the following:
\begin{enumerate}
  \item To the best of our knowledge, this is the first work to use uncertainty estimation for deforestation detection and secondary vegetation mapping using satellite images.
  \item Introduction of a novel uncertainty estimation approach combining transformer-based networks for semantic segmentation with state-of-the-art uncertainty estimation techniques.
  \item Introduction of an operational methodology to aid the annotating and auditing process in semantic segmentation tasks, using uncertainty estimation.
  \item Comparison of multiple uncertainty estimation methods including MC Dropout, ensembles and evidential learning for deforestation and secondary mapping applications.
  

% A comparison of multiple uncertainty methods
% Propose an architecture based on transformers 
\end{enumerate}

\section{Organization of the remaining parts of this thesis}

Chapter \ref{sect:RelatetWorks} describes the related work available in the literature for uncertainty estimation using confidence-based approaches, multi-output approaches and evidential learning, and the use of uncertainty maps for aiding the annotating procedure.

Chapter \ref{sect:Fundamentals} provides the fundamental concepts and theory for a better understanding of the proposed method.

Chapter \ref{sect:Methodology} introduces and explains the preliminary proposed method for aiding the annotating and auditing process using MCD and ensembles as uncertainty estimation techniques.

Chapter \ref{sect:Results} presents the datasets employed in the preliminary study, the experimental protocol followed in the experiments and the preliminary results obtained for the different uncertainty methods in deforestation detection.

Chapter \ref{sect:Conclusions} provides directions for the development of the proposed method.

% summarizes the conclusions derived from the performed experiments and 
   \chapter{RELATED WORK}\label{sect:RelatetWorks}

Recent works have used fully convolutional neural networks (CNNs) for deforestation mapping \cite{ortega2021comparison, adarme2022multi, torres2021deforestation}. In those works, the input to the CNNs was the concatenation of the optical images acquired at different dates, i.e., $T_{-1}$ and $T_{0}$, in a so-called early fusion scheme. In \cite{ortega2021comparison}, an encoder-decoder fully convolutional network model based on the ResUnet \cite{jha2019resunet++} delivered the best results when compared to different CNN architectures used for deforestation detection in the Brazilian Amazon. 

Uncertainty in machine learning models can be divided into aleatoric or data uncertainty and epistemic or model uncertainty. Data uncertainty describes the confidence of the data and it is related to the inherent randomness of the input data. Data uncertainty cannot be reduced by increasing the amount of training samples. Model uncertainty describes the confidence of the prediction and it can be reduced by collecting more training data.


% is related to the inability of the model to produce correct outcomes
% it is high when missing training data

Various approaches have been proposed to estimate uncertainty in deep learning models. The first group are confidence-based approaches, which estimate uncertainty directly on the outcome of a single inference run. Multiple works used confidence-based approaches for uncertainty estimation \cite{amorocho1973entropy, wang2014new, zhan2022comparative}. In \cite{wang2014new}, a comparison of confidence-based approaches was made for entropy, maximum margin and least confidence in an active learning setting, where the three approaches performed similarly. 

Alternatively, multiple-outcome uncertainty estimation methods rely on either multiple training runs or multiple inference runs to estimate the uncertainty related to a neural network's weights. These approaches measure the level of disagreement among multiple outcomes for the same input. 

% Multiple-outcome methods estimate the epistemic uncertainty.

% These approaches can only estimate data uncertainty.

The first multiple-outcome approach trains an ensemble of networks and computes statistical measures that consider the different predictions of the individual networks that compose the ensemble \cite{lakshminarayanan2017simple, mehrtash2020confidence}. Training multiple models is, however, computationally expensive. 

Bayesian networks learn the posterior distribution for a network's trainable weights, allowing to compute the principled predictive uncertainty. However, Bayesian techniques have been proven to be unpractical for deep neural networks due to the large amount of data needed, proportional to the number of network parameters \cite{gawlikowski2021survey}. 

The most common approximation for Bayesian networks is Monte Carlo Dropout (MCD) \cite{gal2016dropout}. Dropout is commonly used during training as a regularization technique. MCD additionally uses dropout at inference, producing a different outcome in each inference run. Uncertainty is then estimated by calculating statistical measures like variance and entropy over a predefined number of inference runs.

% Multiple works have used MCD in classification problems \cite{}.
% The most widely adopted approximation of Bayesian networks is MCD. Multiple works have used MCD in classification problems 
% have also been used for uncertainty estimation. Bayesian networks learn the probability distribution for all trainable weights. 
% Each inference run will draw a different outcome from the learned probability distributions. Uncertainty is estimated as a statistical measure applied to multiple inference runs. 
Recently, various works have used MCD for uncertainty estimation in semantic segmentation applications that rely on deep learning models. Many of those works used U-Net \cite{ronneberger2015u} based networks, and employed MCD at inference time to approximate a Bayesian network. Such an approach has been employed in many application areas such as urban mapping with aerial and satellite images \cite{dechesne2021bayesian}, medical image segmentation \cite{nguyen2021comparison, kwon2020uncertainty}, and fingerprint ROI segmentation \cite{joshi2021explainable}. MCD was also used in \cite{huang2018efficient} to estimate uncertainty in the semantic segmentation of video frames. In all the above-mentioned works, however, the estimated uncertainty maps were only used in the analysis of the corresponding models' predictions, i.e., they were not employed in further processing steps.%, and they did not propose any further procedure to take advantage of uncertainty in the auditing procedure. 

In \cite{wu2022closer}, the resulting uncertainty map estimated using MCD was used as an additional input to train a second U-Net network. The map was concatenated with the original input image, for the semantic segmentation of scanned historical maps. % Although that work used uncertainty to improve the classification results, it did not involve any human auditing based on the uncertainty maps. 
Multiple-outcome methods are computationally expensive because they require to either train or infer multiple times. Evidential deep learning \cite{sensoy2018evidential} is an alternative which estimates model uncertainty with a single training and inference run. It assumes that the outcome of a single deterministic network is a subjective opinion and learn the function leading to those opinions as a Dirichlet distribution, by directly estimating the Dirichlet parameter $\alpha$ at the output of the network. Their work performed similarly to multiple-outcome approaches while being less computationally expensive. 

Recent works have used evidential learning for semantic segmentation \cite{do2022epistemic, holmquist2023evidential, tong2021evidential}. In \cite{do2022epistemic}, a FCN is used for evidential learning by replacing the softmax layer with ReLU and attaching a Dirichlet layer at the outcome, which produced the Dirichlet parameters $\alpha$ at the output. They estimated uncertainty for the semantic segmentation of underwater imagery. Their approach was the same as used in this work, although the application field was a different one. Although they qualitatively observed a correspondence between low accuracy and high uncertainty, they did not quantify the benefits of using uncertainty. In \cite{holmquist2023evidential}, evidential learning was used for class incremental learning, where a previously trained network is extended with new classes. The authors modeled the occurrence of an unknown class (background) as the estimated uncertainty. The proposed approach outperformed other state-of-the-art methods for incremental learning. In \cite{tong2021evidential}, 


In a way similar to our work, multiple methods have used confidence-based approaches, MCD, ensembles and evidential learning for uncertainty estimation in an active learning scheme, although they were devised for different application fields \cite{zhdanov2019diverse, di2019deep, gal2017deep, kirsch2019batchbald, aghdam2019active, ren2021survey, yang2017suggestive, hemmer2022deal}. In \cite{zhdanov2019diverse}, a confidence-based approach (maximum margin) was used to select the most relevant samples in Natural Language Processing (NLP) and image classification tasks. The authors did not assess other uncertainty estimation methods. In \cite{di2019deep}, the authors inferred on test data, estimated the uncertainty maps, and the samples with highest uncertainty were sent to an expert who manually annotated them. Then the network was re-trained and the cycle was repeated for multiple iterations. That work was employed in the semantic segmentation of histology data. Similarly, \cite{gal2017deep} used MCD to compute uncertainty metrics for active learning, in the context of image classification; the proposed method was evaluated on the MNIST dataset and biomedical data. In \cite{aghdam2019active}, MCD was used to compute a guiding metric for active learning in object detection tasks, by annotating and fine-tuning the network with the highest ranked test samples according to their uncertainty scores. They evaluated the method using pedestrian detection datasets. In \cite{yang2017suggestive}, ensembles were used for selecting images with the highest uncertainty to be re-annotated in an active learning scheme for biomedical image segmentation. Different from our work, they selected a specific number of images with the highest uncertainty for an auditor to annotate. Instead, we select image regions within a remote sensing raster for an auditor to re-annotate. In \cite{hemmer2022deal}, evidential deep learning was used for active learning. Their approach outperformed multiple-outcome approaches like MCD and ensembles. However, they only did experiments with image classification. Instead, our work estimates uncertainty for semantic segmentation. 

In \cite{zhan2022comparative}, a comparison of multiple uncertainty estimation techniques including confidence-based methods, MCD and ensembles was made for active learning in image classification tasks. They found that none of the methods performed significantly better than the others.


\textit{Works using uncertainty in transformers}

Uncertainty has been used with transformer networks working with NLP tasks instead of images. In \cite{pei2022transformer}, uncertainty was estimated using Monte Carlo Droput (MCD), ensembles and a novel uncertainty estimation method for a transformer network in a NLP task. Similarly, \cite{sankararaman2022bayesformer} made modifications to the transformer architecture by applying independent dropout masks to key, query and value vector representations and estimates uncertainty using MCD in language classification tasks. The authors in \cite{vazhentsev2022uncertainty} compared multiple uncertainty estimation techniques including MCD, ensembles and simpler ones like the Mahalanobis distance in a transformer network for NLP tasks. They found that simple methods like Mahalanobis distance performed on par compared to more computationally expensive methods like MCD and ensembles.

Recently, vision transformers have proved to be a relevant alternative to traditional fully convolutional networks for semantic segmentation. In \cite{ranftl2021vision}, the authors used an encoder-decoder architecture where the encoder consisted of a sequence of vision transformers (ViT) \cite{dosovitskiy2020image}, while the decoder consisted of convolution and upsampling operations. However, no works have been published for uncertainty estimation in vision tranformers for semantic segmentation. In \cite{yang2021uncertainty}, the authors estimated uncertainty using a separate Fully Convolutional Network (FCN) called uncertainty quantification network, and then used the uncertainty outcomes as a guiding input for ViT. However, they did not estimate the ViT uncertainty and their work did not consider semantic segmentation. In \cite{bin2022efficient} the Multi-Head Attention (MHA) from ViT was replaced by Fourier attention consisting of a 2D fast Fourier transform \cite{lee2021fnet}, which was more computationally efficient and presented comparative effectiveness to MHA. Uncertainty was estimated using MCD on the damage assessment head, which was composed by convolution operations. They applied their method for semantic segmentation of disaster assessment in aerial images. However, uncertainty was not measured at the attention stage.


   \chapter{FUNDAMENTALS}\label{sect:Fundamentals}
This chapter presents the theoretical foundations on which the proposals presented in subsequent chapters are based. It is organized into two sections.
The first section presents the literature's most widely used uncertainty assessment approaches. The second section briefly introduces the so-called Visual Transformers.

\section{Uncertainty Assessment Approaches}

We group uncertainty estimation into 
\begin{itemize}
    \item confidence-based, 
    \item multiple-outcome-based, and
    \item evidential learning,
\end{itemize}
\noindent 
methods as detailed in the following.

\subsection{Confidence-Based Approaches}

The first group of methods depends on the model's confidence in its prediction. Below, we briefly present the most widely used metrics computed from the discrete probability distribution at a model's output.

In the following, we denote with $\textbf{y}=[ y_1,..., y_K]$ the $K$-dimensional vector representing the discrete probability distribution assigned by the model for a given input. It is assumed that $\textbf{y}$ is a simplex, i.e., $y_k \geq 0$ for all $k \in \{1,...,K\}$ and that $\{y_k\}_{k=1}^K$ add up to one.

\bigskip
\noindent
\textit{Entropy}:%\subsubsection{Entropy}

The most widely used metric for classifier's confidence is the entropy defined as

\begin{equation} \label{eq:Entropy}
h(\textbf{y}) = - \sum_{k} y_k*log (y_k)
\end{equation}

The highest model confidence corresponds to the lowest entropy value ($h(\textbf{y})=0)$ when the model assigns probability equal to one to one particular class and probability zero to all other classes. Conversely, the highest entropy value is reached when the model gives equal probabilities to all classes, corresponding to the lowest confidence.%The highest model confidence corresponds to the lowest entropy value ($h(\textbf{y})=0)$ when the model assigns probability equal to one to one particular class, and probability zero to all other classes. The highest entropy value is reached when the model assigns equal probabilities to all classes, which corresponds to the lowest confidence.


\bigskip
\noindent
\textit{Maximum Margin}:


Margin-based confidence \cite{makingsynthesis} corresponds to the difference between the highest and the second most likely class, formally:

\begin{equation}
MaxMargin(\textbf{y}) = y_a - y_b
\end{equation}
\iffalse
\begin{equation}
u_i = P(\hat{y}_{1,i}|x_i)-P(\hat{y}_{2,i}|x_i)
\end{equation}
\fi

\noindent
where $a$ and $b$ are the first and the second most likely class labels, respectively. 
%$s_i$ is the margin-based uncertainty for pixel location $i$, $P(y_{1,i})$ and $P(y_{2,i})$ are the highest and second highest class probabilities, respectively.

% netzer2011reading
\bigskip
\noindent
\textit{Most Likely Label}:

In this approach, one takes as confidence the raw predicted probability value of the most likely class label. 
\begin{equation}
MaxL (\textbf{y}) = max ({ \{ y_k \}_{k=1}^K })
\end{equation}
\bigskip
Notice that $MaxL (\textbf{y})$ ranges from $1/K$ to 1.
\iffalse
\begin{equation}
u_i = -P(\hat{y}_{n,i}|x)
\end{equation}
\fi

%Where $P(\hat{y}_{n,i}|x)$ is the predicted class probability for the class with the least confidence $n$ and pixel location $i$. 

% \subsection{Maximum Mutual Information}

% This method calculates the Mutual Information (MI) as uncertainty metric. The samples with maximum MI correspond to the most uncertain.

% \begin{equation}
% s_i = H[y|x]-
% \end{equation}


\subsection{Multiple-Outcome Approaches}

The methods introduced in the previous subsection are simple to implement and require a single training and inference run. Differently, the consistency-based uncertainty estimates rely on outcomes of multiple inference runs corresponding by either considering multiple random initialization values during training (e.g., ensembles), or creating randomness at inference (e.g., Monte Carlo Dropout). In the following, we introduce the consistency-based methods.
% , or using the same network on augmented versions of the same input data (Test-Time Augmentation)

To accommodate multiple outcomes for the same input, we denote hereafter with $\textbf{y}^{(i)}=[ y_1^{(i)},..., y_K^{(i)}]$ the $K$-dimensional vector representing the discrete probability distribution assigned by the $i$-th model for a given input, where $i \in \{1,...,n\}$, and $n$ is the number of outcomes.

The so-called final prediction is given by the average probability vector computed over the $n$ outcomes, formally:
 \begin{equation} \label{eq:FinalPrediction}
 \boldsymbol{\mu}
 =\frac{1}{n}\sum_{i=1}^{n}{\textbf{y}^{(i)}}
 \end{equation}

\bigskip
\noindent
\subsubsection{Monte Carlo Dropout (MCD)}

Dropout is a technique commonly used in the training phase to reduce overfitting. All the forward and backward connections with a dropped node are temporarily removed, thus creating a new network architecture out of the parent network. The set of dropped-out nodes is selected randomly at each forward/backward pass. Usually, at inference time all nodes remain active. 

MCD differs from the conventional Dropout because the strategy is applied in the inference step. In MCD, the inference is carried out $n$ times, each time with a different set of dropped-out neurons, resulting in a different outcome at each run. 
 It can be proved that this procedure is an approximation of Bayesian inference \cite{gal2016dropout}. It corresponds to sampling from the posterior distribution in each inference run.  %Using Dropout at inference is equivalent to sampling from the posterior distribution in each inference run, and it can approximate Bayesian inference \cite{gal2016dropout}. In MCD, the inference is performed for a number $n$ of times using Dropout, resulting in a different outcome at each run. 

The uncertainty value is obtained by calculating uncertainty metrics over the $n$ predictions. This calculation is carried out in semantic segmentation for each image pixel, producing an uncertainty map. In Subsection \ref{ref:metrics}, we describe different ways to compute uncertainty metrics from these $n$ results.


\bigskip
\noindent
\subsubsection{Ensembles}

Ensemble methods combine the predictions of several different deterministic networks at inference. A key issue in designing network Ensembles  is diversity. In other words,  the errors of each ensemble member should concentrate on different regions of the feature space.

Diversity can be achieved in different ways. In Deep Learning, it is common practice to create ensembles by training the same basic network architecture starting from different random initializations \cite{lakshminarayanan2017simple}. So, each training run produces a different ensemble member. 

This is the strategy adopted in our research. Therefore, we obtain $n$ distinct networks, obtaining $n$ different results for each input. 
 As with the Dropout strategy, uncertainty is calculated upon the multiple outcomes, using different metrics, as described in Subsection \ref{ref:metrics}.


It is worth noticing that this method is more computationally expensive than MCD, because it involves training $n$ networks, whereas MCD requires training a single network. %Ensembles consist of training the same network architecture multiple times. Due to random initialization, each training run produces different outcomes at inference. ensembles have been successfully used to boost performance and also to estimate uncertainty \cite{lakshminarayanan2017simple}. Uncertainty is obtained by calculating metric values over the inference outcomes of separately trained networks. This method is more computationally expensive compared to MCD, which only needs to train once. The uncertainty metrics we considered are described in Subsection \ref{ref:metrics}.

% As in MCD, we assessed multiple uncertainty metrics: Predictive entropy, predictive variance, mutual information and expected Kullback-Leibler divergence.


\subsubsection{Uncertainty Metrics}
\label{ref:metrics}

This subsection describes the metrics used in this research to estimate the uncertainty from the $n$ results produced by MCD, Ensembles, or TTA.

\bigskip
\noindent
\textit{Predictive Entropy}: 

In few words, the Predictive Entropy $H(\textbf{y})$ is the \textit{entropy}, as defined in eq. \ref{eq:Entropy},  computed upon the \textit{final prediction} defined in eq. \ref{eq:FinalPrediction}  \cite{wang2019aleatoric, mcclure2019knowing}. So, the definition of predictive entropy is given by:

\[H(\textbf{y})\approx -\frac{1}{k}\sum_{k=1}^{K}{\mu_k}log({\mu_k}) }\]

%\[H(\textbf{y}(\textbf{p})|\textbf{x}(\textbf{p}))\approx -(1/K)*\sum_{k=1}^{K}{\mu_k(\textbf{p})log(\mu_k(\textbf{p})) }\]

\noindent
where $\mu_k$ is the mean probability across inference runs for class $k$, and $K$ is the number of classes. Additionally, we assessed other uncertainty metrics as an ablation study. %We compared the aforementioned predictive entropy with three other uncertainty metrics: Predictive variance, Mutual Information (MI) and Expected Kullback-Leibler Divergence. These 3 metrics compute the divergence between the average prediction $\mu_k(\textbf{p})$ and the individual predictions.

 

\bigskip
\noindent
\textit{Predictive Variance}: 

The Predictive Variance is the average variance computed over the $K$ classes \cite{yang2017suggestive}. 
Recall that the $k$-th component $\mu_k$ of the final prediction vector  $\boldsymbol\mu$ (eq. \ref{eq:FinalPrediction}) is given by


\[\mu_k= \frac{1}{n}\sum_{i=1}^{n}{y_k^{(i)}

So, the variance of the $n$ predictions for class $k$ can be computed as
\[\sigma_k^2= \frac{1}{n}\sum_{i=1}^{n}{(y_k^{(i)}-\mu_k)^2

The Predictive Variance $\boldsymbol\sigma^2$ is the average of class variances, specifically:
\[\boldsymbol\sigma^2= \frac{1}{K}\sum_{k=1}^{K}{\sigma_k^{2}}
%^2(\textbf{p}) }\]
%\[u(\textbf{p})= \frac{1}{k})*\sum_{k=1}^{K}{\sigma_k^2(\textbf{p}) }\]


\bigskip
\noindent
\textit{Mutual Information}: 

 The Mutual Information $MI(\textbf{y})$ assesses the similarity between a set of random variables that were sampled simultaneously. It informs how much information from one random variable is present in another \cite{learned2013entropy}. Its value is the difference between the predicted entropy computed on the final prediction and the average of the entropy of each prediction.

\[MI(\textbf{y}})=H(\textbf{y}) - \frac{1}{nK}\sum_{i=1}^{n}{\sum_{k=1}^K y_k^{(i)}log (y_k^{(i)})}\]
%[MI(\textbf{y}|\textbf{x})=H(\textbf{y}|\textbf{x}) - (1/n)*\sum_{i=1}^{n}{H^{(i)}(\textbf{y}^{(i)}|\textbf{x}^{(i)})}\]

\bigskip
\noindent
\textit{Expected Kullback-Leibler Divergence}: 

This Kullback-Leibler (KL) Divergence measures the divergence between two probability distributions \cite{kullback1951information}. In this context, the Expected KL Divergence (EKL) measures the average (expected) KL divergence between the final and the individual predictions, formally:

\[EKL(\textbf{y})= -\frac{1}{n}\sum_{i=1}^{n}{\sum_{k=1}^{K}{\mu_k log(\mu_k/y_k^{(i)}) }}\]



\section{Evidential learning}

Multiple-outcome approaches are computationally expensive due to the need of either training or inferring multiple times. Evidential learning is an alternative uncertainty estimation method which allows to obtain uncertainty using a single training run and a single inference step \cite{sensoy2018evidential}. 

Let's assume that each class elected by one of the aforementioned multi-output approaches represents a sample drawn from some underlying categorical distribution parametrized by a probability vector \boldsymbol{p}. Intuitively, if one of those classes often comes about, this is an indication of low uncertainty. On the other hand, if more than one class occur with similar frequency, this is a strong indication of high uncertainty.

Going a step further, we assume that each probability vector \boldsymbol{p} is drawn from some other higher-order distribution.

Evidential learning seeks to learn the parameters of such high-order distributions. So, instead of drawing samples from distributions, evidential learning aims at learning the distribution over these distributions, called hereafter evidential distribution, directly from the data. Therefore, a sample drawn from the evidential distribution defines itself a distribution over the data.

Let's assume that a class label $L \in \{1,...,k\}$ is a sample drawn from a likelihood function of the categorical form parameterized by probability vector \boldsymbol{p},  i.e.,

\[ L \sim  categorical(\boldsymbol{p})\]

As in \cite{sensoy2018evidential}, we further assume that the discrete probability distribution \boldsymbol{p} can be estimated using a Dirichlet prior:
%[??]
\[\boldsymbol{p} \sim Dirichlet(\boldsymbol{\alpha)}\]
The Dirichlet prior is itself parametrized by a set of $K$ (the number of classes) parameters, here denoted $\boldsymbol{\alpha}$. A sample drawn from the Dirichlet distribution is a realization of the distribution probabilities \boldsymbol{p}.










%-----------------

Typical neural networks have a softmax function at the last layer to produce class probabilities $\{{y_k}\}_{k=1}^K$ from the logits at its inputs produced by the network, as shown in Figure \ref{fig:cnn}. 

\begin{figure*}[h!] 
\centering
\includegraphics[scale=0.55]{figures/2-Fundamentals/Typical CNN.png}
\caption{Typical CNN architecture with a softmax attached to the last layer.}
\label{fig:cnn}
\end{figure*}

In evidential network, instead, the network computes the so-called evidences $e_k$, which are applied to the last layer to compute the parameters of the Dirichlet distribution $\boldsymbol{\alpha}=[\alpha_1,...,\alpha_K]$, as $\alpha_k = e_k + 1$, rather than a single probability distribution $\boldsymbol{y}=[y_1,...,y_K]$ (see Figure \ref{fig:beliefs}).

\begin{figure*}[h!] 
\centering
\includegraphics[scale=0.55]{figures/2-Fundamentals/Evidential beliefs.png}
\caption{Evidential approach: the CNN computes the evidences from which the Dirichlet parameters are computed. The belief mass and the uncertainty are directly computed from the Dirichlet parameters.}
\label{fig:beliefs}
\end{figure*}

The uncertainty and class probabilities can be directly computed from the Dirichlet parameters as follows.

Let $b_k$, be the belief mass for class $k \in \{1,...,K\}$ and $u$ the uncertainty mass, where

\[u+\sum_{k=1}^{K}b_k=1\] 

\noindent
with $u\geq 0$ and $b_k \geq 0$. The belief mass can be computed from the evidence $e_k$ for class $k$ (provided from the network output), as 

\[b_k=\frac{e_k}{S}\] 
\[u=\frac{K}{S}\] 

\noindent
where $S=\sum_{k=1}^{K}(e_k+1)$.

Additionally,  the expected probability for the $k$-th class is the mean of the corresponding Dirichlet distribution and computed as


\[ y_k= \frac{\alpha_k}{S} \]
% $\textbf{\alpha}=[\alpha_1,...,\alpha_K]$ %


% During training, the lossf unction maximizes model fit and also minimizes incorrect evidence.
The equations above are easily differentiable and bring no difficulty to the backpropagation of the error gradient.
The loss function for training the network is composed of up to three terms that take into account the cross-entropy to minimize the prediction error, the variance of the Dirichlet experiment generated by the neural network and the Kullback-Leibler (KL) divergence as a regularization term.

% by penalizing the values from the "not known" state that do not contribute to data fit\textcolor{red}{what else Jorge?}.

As indicated at the beginning of this explanation, this approach involves training a single network and a single inference step to compute class probability distributions and uncertainty.

   \chapter{UNCERTAINTY ESTIMATION TO REDUCE THE MANUAL ANNOTATION EFFORT }\label{sect:Methodology}
This chapter describes the design of experiments to evaluate the impact of uncertainty estimation to aid the annotation and auditing process in deforestation mapping in the Amazon rainforest, using Monte Carlo Dropout (MCD), ensembles, and a single confidence measure, i.e., the entropy. The next chapter reports and discusses the results of these preliminary experiments. 

The following sections describe the adopted methodology, which includes adding the distance from the past deforestation maps to the input data. Besides, in a second protocol, the training relies only on labeled training data from earlier dates avoiding manual labeling of parts of the target image for training the network.%This chapter describes experiments the proposed method based on uncertainty estimation to aid the annotating and auditing process in deforestation mapping, using Monte Carlo Dropout (MCD) and ensembles to obtain the uncertainty map. Preliminary results are presented in the following chapters for this methodology. In the following, we also describe the use of an additional input feature to improve the classification outcomes, namely the distance to the past deforestation map. Besides, we propose a protocol to train using images from the past and inferring on a new upcoming date, which allows the proposed method to be used in an real-world operational setting.
% This chapter describes the proposed method based on uncertainty estimation to aid the annotating and auditing process, using Monte Carlo Dropout (MCD) and ensembles to obtain the uncertainty map. In the following, we also describe the use of an additional input feature to improve the classification outcomes, namely the distance to the past deforestation map. Besides, we propose a protocol to train using images from the past and inferring on a new upcoming date, which allows the proposed method to be used in an real-world operational setting.


\section{ResUnet-Dropout network}
The architecture of the network used in this work is described in Figure \ref{fig:resunet}. It is a modified version of the fully convolutional ResUnet used in \cite{ortega2021comparison}. Dropout was added in each up-sampling layer following recent works \cite{dechesne2021bayesian, nguyen2021comparison, kwon2020uncertainty}. We used spatial dropout in all cases \cite{lee2020revisiting}. To detect the deforestation changes from time $T_{-1}$ to $T_0$, the input to the network is the concatenation of Sentinel-2 images corresponding to both dates $(S2_{T_{-1}}:S2_{T_{0}})$, where $T_0$ is the current date, $T_{-1}$ is the date from one year before the current date, $S2_{T_{0}}$ is the Sentinel-2 image for the current date, and $S2_{T_{-1}}$  is the Sentinel-2 image for the previous year.

\begin{figure*}[h!]
\centering
% \includegraphics[scale=0.6]{figures/method/model.png}
\includegraphics[scale=0.5]{figures/3-Methodology/resunet network.pdf}
\caption{Fully convolutional ResUnet architecture. Dropout is enabled during training and inference. Adapted from \cite{ortega2021comparison}}
\label{fig:resunet}
\end{figure*}

\section{Temporal distance to past deforestation}

Deforestation usually occurs spatially adjacent to regions deforested in previous years. In this work, we leverage from the information about past deforestation taken from PRODES~\cite{prodes}, and use it as an additional input feature map, which is concatenated to the input Sentinel-2 image pair. We define the temporal distance to past deforestation map $D(x,y,T_0)$ as:

\[D_t(x,y,T0) = T_0 - deforestation\_year(x,y) + 1\]

\noindent
where $T_0$ is the current year, and $deforestation\_year(x,y)$ is the year of deforestation for the pixel at location ($x,y$). We set $deforestation\_year(x,y)$ to $T_0$ in pixel locations where deforestation has not occurred. If we trained and tested on the most recent image pair, the input tensor comprised $(S2_{T_{-1}}, S2_{T_{0}}, D_{T_{-1}}(x,y))$, where $T_0$ and $T_{-1}$ correspond to the current and previous year acquisition dates, and $S2_{T_{0}}$ and $S2_{T_{-1}}$ represent the corresponding Sentinel-2 images for $T_0$ and $T_{-1}$ dates, respectively.

\section{Uncertainty to Aid the Auditing Process}


As mentioned, the experiments used as ground truth were the deforestation reports published by PRODES. In PRODES's methodology, 100\% of the non-deforested area until the current year is visually inspected and audited by INPE's personnel in each upcoming year. In this work, we propose an alternate semi-automatic workflow where we first train a neural network on image pairs from previous years and then run the inference on a recent image pair (as explained in Section \ref{sec:training_past_dates}). 

So, we consider a methodology to reduce the labeling/auditing effort that involves two steps. 
The first step consists of running an automatic classification using a deep learning model trained as outlined before. Beyond delivering a deforestation map, the first step computes the uncertainty associated with the classification of each pixel. 
In the second step, the photo interpreter only audits the areas whose uncertainty computed in step 1 exceeds a user-selected threshold. As classification metrics, we obtain $F_{1_{low}}$ for samples with low uncertainty, and $F_{1_{high}}$ for samples with high uncertainty. We call the percentage of test samples with high uncertainty the Alert Area (AA). We expect that $F_{1_{low}}>F_1>>F_{1_{high}}$, where $F_1$ refers to classification metrics prior to applying the uncertainty methodology.

\bigskip
The final deforestation report results from joining:
\begin{itemize}
\item on the pixels with low uncertainty, the classes assigned by the model

\item on the pixels with high uncertainty, the classes assigned by the photo-interpreter 
\end{itemize}

\bigskip

$F_{1_{audit}}$ represents the classification metrics after the expert auditor has correctly re-annotated the samples with high uncertainty corresponding to the AA percentage.

%The experiments reported next aimed to test the following two hypotheses:


%1) the accuracy on the areas with low uncertainty meets the application performance requirements, and


%2) Audited by human analysts, the pixels with high uncertainty will make up a small portion of the image, thus reducing the human annotation effort significantly.

\iffalse
Reference ground truth for deforestation detection in the Amazon is acquired yearly following the PRODES methodology~\cite{prodes}. In such methodology, 100\% of the Amazon area outside of the past deforestation regions is visually inspected and audited by INPE's personnel in each upcoming year. In this work, we propose an alternate semi-automatic workflow where we first train a neural network on image pairs from previous years, and then run the inference on a recent image pair (as explained in Section \ref{sec:training_past_dates}). Instead of auditing 100\% of total area, we propose to estimate the classification uncertainty for each pixel location. %, which we expect to produce low uncertainty values for areas we can trust and have high classification metrics, and high uncertainty values for areas the network does not know if it is correct, where we expect classification metrics to be significantly lower. 
Then, an uncertainty threshold should be applied on the uncertainty map, and compute the classification metrics for samples with low and high uncertainty. Namely, we obtain $F_{1_{low}}$ for samples with low uncertainty, and $F_{1_{high}}$ for samples with high uncertainty. We call the percentage of samples with high uncertainty as the Alert Area (AA). 

We expect that for a low AA, the majority of inferred samples would represent a high $F_{1_{low}}$, while a minority of the inferred samples corresponding to the AA percentage produce a significantly lower $F_{1_{high}}$ such that $F1_{low}>F_1>>F1_{high}$, where $F_1$ refers to classification metrics prior to applying the uncertainty methodology. Finally, instead of auditing 100\% of the image, only samples with high uncertainty, corresponding to a small percentage (AA), would be sent to an expert auditor who would create their corresponding reference ground truth.  
\fi

\section{Uncertainty Methods}

For the preliminary results, a comparison of MCD and ensembles is presented for the ResUnet base network. Besides, entropy from a single forward run is also assessed. In the latter case, we calculated entropy from $n$ different training runs and present the best and worst case in terms of $F_{1_{low}}$.
 

\section{Training On Samples From the Past To Infer On Upcoming Date}
\label{sec:training_past_dates}
Recent works in deforestation detection used an image from the same date to train and test, by dividing the raster into non-overlapping tiles and setting a percentage of those tiles to train and test within the same image \cite{ortega2021comparison}. However, in a real-world application, collecting ground truth data for each upcoming date would be costly and unfeasible. In this work, we propose an operational solution where we train the network a date pair from the past and infer on a new upcoming date which was not seen during training. Formally, we trained on $(S2_{T_{-2}}, S2_{T_{-1}}, D_{T_{-2}}(x,y))$ and inferred on $(S2_{T_{-1}}, S2_{T_{0}}, D_{T_{-1}}(x,y))$, where $S2$ corresponds to a Sentinel-2 image. $T_0$ corresponds to the image from the current date, $T_{-1}$ to the previous year date, and $T_{-2}$ to the second year before $T_0$. In this way, the proposed approach could directly be applied in a real-world application. % The aforementioned training scheme is presented in Figure \ref{fig:resunet_past_dates}.

% Recent works in deforestation detection used an image from the same date to train and test, by dividing the raster into non-overlapping tiles and setting a percentage of those tiles to train and test within the same image \cite{ortega2021comparison}. However, in a real-world application, collecting ground truth data for each upcoming date would be costly and unfeasible. In this work, we propose an operational solution where we train the network on date pairs from the past and infer on a new upcoming date which was not seen during training. Formally, we trained on [$(S2_{T_{-2}}, S2_{T_{-1}}, D_{T_{-2}}(x,y))$, $(S2_{T_{-3}}, S2_{T_{-2}}, D_{T_{-3}}(x,y))$, $(S2_{T_{-4}}, S2_{T_{-3}}, D_{T_{-4}}(x,y)),...$] and inferred on $(S2_{T_{-1}}, S2_{T_{0}}, D_{T_{-1}}(x,y))$, where $S2$ correspsponds to a Sentinel-2 image. $T_0$ corresponds to the image from the current date, $T_{-1}$ to the previous year date, and so on. In this way, the proposed approach could directly be applied in a real-world application. The aforementioned training scheme is presented in Figure \ref{fig:resunet_past_dates}.

% \begin{figure}[h!]
% \centering
% \includegraphics[scale=0.45]{figures/method/past_dates_scheme.png}
% \caption{Network scheme training on samples from past date pairs, and inferring on an unseen date.}
% \label{fig:resunet_past_dates}
% \end{figure}
   \chapter{PRELIMINARY STUDY: EXPERIMENTAL ANALYSIS}\label{sect:Results}

This chapter reports the experiments carried out in order to validate the method proposed in the previous chapter as a preliminary study. First, the datasets used in the experiments for deforestation detection are presented. Then, the experimental protocol followed for the proposed  methodology is described, and the parameter setup is detailed. Finally, the results obtained in the experiments are reported.

\section{Study Areas}\label{sect:Datasets}

We evaluated the proposed method in two study areas in the Brazilian Legal Amazon (Figure \ref{fig:study_areas}). Both sites have a mixed land cover. The first site is located in the Para state (PA), with an area of $92\times 177 Km^2$. This site is mainly composed of dense evergreen forest and pasture. For the single date pair experiments, we used $[T_{-1}, T_0]=[2018, 2019]$. For the experiments training with past dates, we trained with the $[T_{-2}, T_{-1}]=[2017, 2018]$ date pair and tested on a new upcoming image using $[T_{-1}, T_0]=[2018, 2019]$. The second site is located in the Mato Grosso state (MT), with an area of $134\times 208 Km^2$. This site is mainly composed of dense forests, soy fields and pastures. We used $[T_{-1}, T_0]=[2019, 2020]$. For the multiple date pair experiments, we trained with the $[T_{-2}, T_{-1}]=[2018, 2019]$ date pair. In all cases, we used Sentinel-2 images and we did not use the 60$m$ bands. We used the remaining channels with $10m$ resolution, by up-sampling the $20m$ bands with the nearest neighbor method.

\begin{figure*}[ht!]
    \centering
		\includegraphics[scale=0.4]{figures/4-Experimental/study_areas_v2.png}
	\caption{Geographical location of the study areas, and RGB composition of the corresponding Sentinel-2 images acquired at $T_{-1}$.}
\label{fig:study_areas} % to-do: improve blocks?
\end{figure*}



\section{Experimental Protocol}
We extracted overlapping sub-images of size $128\times 128$ for training with 70\% overlap. We selected only the sub-images with at least 2\% of the deforestation class for training. The train, validation and test areas were selected by splitting the site into non-overlapping tiles and selecting 40\% for training, 10\% for validation and 50\% for testing. We used on-line data augmentation by randomly applying rotations and horizontal and vertical flips on each training batch.

We used the same parameter configuration as in \cite{ortega2021comparison} for the ResUnet architecture, and we added dropout in each decoder stage in accordance to previous works \cite{dechesne2021bayesian, nguyen2021comparison, kwon2020uncertainty}. The parameter configuration is presented in Table \ref{tab:parameters}. We used Max-pooling as down-sampling operator, and nearest-neighbor up-sampling. We used dropout rate of 0.25 in all cases. We used batch size 32 and weighted categorical cross entropy with weights of 0.1 for non-deforestation, 0.9 for deforestation and 0 for not considered areas including past deforestation and cloudy regions. For the accuracy assessment, we ignored pixels within a spatial buffer of 2px surrounding the ground truth deforestation polygons, and also pixel clusters predicted as deforestation with an area smaller than $6.25ha$. The reason for the former rule is to avoid misregistration problems, also considering that PRODES references have a spatial resolution equivalent to Landsat multispectral bands. The reason for the later rule was that $6.25ha$ is PRODES minimum mapping unit.

% Please add the following required packages to your document preamble:
% \usepackage{multirow}
\usepackage{adjustbox}
\begin{table*}[ht!]
\centering

\caption{Network architecture for the ResUnet-based FCN. C: Convolution, DS: Down-sampling, RB: Residual block, D: Dropout, US: Up-sampling. Convolution layers are parametrized as ($kernel\_width \times kernel\_height, \#filters$)}
\scalebox{0.8}{
\begin{tabular}{c|cc|c|c}
\hline
\textbf{Encoder} & \multicolumn{2}{c|}{\textbf{Bottleneck}} & \textbf{Decoder} & \textbf{Output}            \\ \hline
DS(RB(3$\times$3, 32))  & \multirow{3}{*}{3$\times$}    & \multirow{3}{*}{RB(3$\times$3, 128)}     & D(US(C(3$\times$3, 128)))  &                            \\
DS(RB(3$\times$3, 32))  &                        &      & D(US(C(3$\times$3, 64)))   & Softmax(C(1$\times$1, $\#classes$)) \\
DS(RB(3$\times$3, 32))  &                        &       & D(US(C(3$\times$3, 32)))   &                            \\ \hline
\end{tabular}}
\label{tab:parameters}
\end{table*}

We used 10 inference runs ($n=10$) for uncertainty estimation using MCD. Correspondingly, we used 10 training runs for uncertainty estimation using ensembles. We compared results for multiple uncertainty metrics and selected the best overall performing metric. For the evidential learning method, which requires a single inference run to produce uncertainty map, we repeated the experiment 10 times and present results for the average values, as well as the worst and best scenarios with respect to $F_{1_{low}}$. 
We present entropy from a single inference run for comparison purposes, where we also present results for the averaged values, along with the worst and best performing single inference run from 10 repetitions, with respect to $F_{1_{low}}$. % Additionally, we quantified the amount of effort required from the expert auditors using our methodology in terms of polygon count. Numerical results correspond to the average of 10 training repetitions in each case. 
Experiments were carried out in an NVIDIA RTX 2080 Ti GPU.  




\section{Results}
\label{sec:results}
% In this section, results are presented for each site individually and then 

In Subsections \ref{sec:results_uncertainty_PA} and \ref{sec:results_uncertainty_MT}, we present results for the proposed methodology PA and MT study areas, respectively. In Subsection \ref{sec:results_polygon}, we present a polygon analysis which attempts to better understand the auditing effort.

% \subsection{Classification and Uncertainty Estimation Results}



\subsection{Uncertainty Estimation Results in PA Site}\label{sec:results_uncertainty_PA}


\begin{table*}[ht!]
\centering
\caption{Uncertainty results for PA site training and testing in [2018, 2019]. $AA=3\%$. First, second and third best results in terms of $F_{1_{low}}$ are highlighted in \textcolor{red}{\textbf{red}}, \textcolor{blue}{\textbf{blue}} and \textcolor{teal}{\textbf{green}} respectively}  % . Train date: [2018, 2019]. Test date: [2018, 2019]. 
\begin{tabular}{c|c|c|ccc}
\hline
\textbf{Method}             & \textbf{Uncertainty Metric} & \textbf{$F_1$}                    & \textbf{$F_{1_{low}}$} & \textbf{$F_{1_{high}}$} & \textbf{$F_{1_{audit}}$} \\ \hline
\multirow{4}{*}{MCD} & Predictive Entropy          & \multirow{4}{*}{84.1}          & 95.3            & 61.0             & 96.9              \\
                            & Predictive Variance         &                                & 90.9            & 77.5             & 96.2              \\
                            & Mutual Information          &                                & 88.2            & 34.0             & 89.3              \\
                            & Expected KL                 &                                & 87.2            & 26.7             & 88.0              \\ \hline
\multirow{4}{*}{Ensemble}   & Predictive Entropy          & \multirow{4}{*}{\textcolor{blue}{\textbf{85.8}}} & \textcolor{blue}{\textbf{96.9}}   & \textcolor{blue}{\textbf{62.3}}    & \textcolor{blue}{\textbf{97.9}}     \\
                            & Predictive Variance         &                                & \textcolor{teal}{\textbf{95.6}}            & \textcolor{teal}{\textbf{72.3}}             & \textcolor{teal}{\textbf{97.5}}              \\
                            & Mutual Information          &                                & 92.2            & 31.3             & 93.1              \\
                            & Expected KL                 &                                & 90.9            & 25.6             & 91.7              \\ \hline
\multirow{2}{*}{Single run} & Entropy (Worst)             & 83.4                           & 94.1            & 66.7             & 96.5              \\  
                            & Entropy (Best)              & \textcolor{red}{\textbf{87.4}}                           & \textcolor{red}{\textbf{97.0}}            & \textcolor{red}{\textbf{74.7}}             & \textcolor{red}{\textbf{98.3}}              \\ \hline

\end{tabular}
\label{tab:PA_results_t1}
\end{table*}



\begin{table*}[ht!]
\centering
\caption{Uncertainty results for PA site training in [2017, 2018] and testing in [2018, 2019]. $AA=3\%$. First, second and third best results in terms of $F_{1_{low}}$ are highlighted in \textcolor{red}{\textbf{red}}, \textcolor{blue}{\textbf{blue}} and \textcolor{teal}{\textbf{green}} respectively}  
\begin{tabular}{c|c|c|ccc}
\hline
\textbf{Method}             & \textbf{Uncertainty Metric} & \textbf{$F_1$}                    & \textbf{$F_{1_{low}}$} & \textbf{$F_{1_{high}}$} & \textbf{$F_{1_{audit}}$} \\ \hline
\multirow{4}{*}{MCD} & Predictive Entropy          & \multirow{4}{*}{\textcolor{teal}{\textbf{78.0}}}          & \textcolor{teal}{\textbf{92.0}}            & \textcolor{teal}{\textbf{60.3}}             & \textcolor{teal}{\textbf{96.0}}              \\
                            & Predictive Variance         &                                & 84.8            & 73.7             & 95.0              \\
                            & Mutual Information          &                                & 86.0            & 34.2             & 88.6              \\
                            & Expected KL                 &                                & 84.6            & 29.4             & 87.0              \\ \hline
\multirow{4}{*}{Ensemble}   & Predictive Entropy          & \multirow{4}{*}{\textcolor{blue}{\textbf{81.4}}} & \textcolor{blue}{\textbf{94.4}}   & \textcolor{blue}{\textbf{63.9}}    & \textcolor{blue}{\textbf{97.2}}     \\
                            & Predictive Variance         &                                & 91.7            & 72.8             & 96.6              \\
                            & Mutual Information          &                                & 90.7            & 46.6             & 93.3              \\
                            & Expected KL                 &                                & 88.5            & 38.7             & 90.8              \\ \hline
\multirow{2}{*}{Confidence} & Entropy (Worst)             & 74.5                           & 83.5            & 52.4             & 89.0              \\
                            & Entropy (Best)              & \textcolor{red}{\textbf{84.9}}                           & \textcolor{red}{\textbf{94.9}}            & \textcolor{red}{\textbf{68.0}}             & \textcolor{red}{\textbf{97.1}}              \\ \hline
\end{tabular}
\label{tab:PA_results_t0}
\end{table*}


\begin{table*}[ht!]
\centering
\caption{Uncertainty results for MT site training and testing in [2019, 2020]. $AA=3\%$. First, second and third best results in terms of $F_{1_{low}}$ are highlighted in \textcolor{red}{\textbf{red}}, \textcolor{blue}{\textbf{blue}} and \textcolor{teal}{\textbf{green}} respectively}  % . Train date: [2018, 2019]. Test date: [2018, 2019]. 
\begin{tabular}{c|c|c|ccc}
\hline
\textbf{Method}             & \textbf{Uncertainty Metric} & \textbf{$F_1$}                    & \textbf{$F_{1_{low}}$} & \textbf{$F_{1_{high}}$} & \textbf{$F_{1_{audit}}$} \\ \hline
\multirow{4}{*}{MCD} & Predictive Entropy          & \multirow{4}{*}{79.2}          & 91.7            & 57.7             & 94.8              \\
                            & Predictive Variance         &                                & 92.2            & 64.0             & 95.9              \\
                            & Mutual Information          &                                & 89.3            & 43.3             & 92.0              \\
                            & Expected KL                 &                                & 87.8            & 39.4             & 90.4              \\ \hline
\multirow{4}{*}{Ensemble}   & Predictive Entropy          & \multirow{4}{*}{\textcolor{blue}{\textbf{81.4}}} & \textcolor{teal}{\textbf{94.1}}            & \textcolor{teal}{\textbf{63.8}}             & \textcolor{teal}{\textbf{96.6}}              \\
                            & Predictive Variance         &                                & \textcolor{blue}{\textbf{95.1}}   & \textcolor{blue}{\textbf{69.0}}    & \textcolor{blue}{\textbf{97.7}}     \\
                            & Mutual Information          &                                & 93.5            & 61.4             & 96.0              \\
                            & Expected KL                 &                                & 93.1            & 60.3             & 95.6              \\ \hline
\multirow{2}{*}{Confidence} & Entropy (Worst)             & 78.6                           & 84.7            & 66.4             & 89.8              \\
                            & Entropy (Best)              & \textcolor{red}{\textbf{83.0}}                           & \textcolor{red}{\textbf{95.3}}            & \textcolor{red}{\textbf{66.2}}             & \textcolor{red}{\textbf{97.3}}              \\ \hline
\multirow{2}{*}{Evidential} & Evidential (Worst)             & 78.6                           & 84.7            & 66.4             & 89.8              \\
                            & Evidential (Best)              & \textcolor{red}{\textbf{83.0}}                           & \textcolor{red}{\textbf{95.3}}            & \textcolor{red}{\textbf{66.2}}             & \textcolor{red}{\textbf{97.3}}              \\ \hline                            
\end{tabular}
\label{tab:MT_results_t1}
\end{table*}


\begin{table*}[ht!]
\centering
\caption{Uncertainty results for MT site training in [2018, 2019] and testing in [2019, 2020]. $AA=3\%$. First, second and third best results in terms of $F_{1_{low}}$ are highlighted in \textcolor{red}{\textbf{red}}, \textcolor{blue}{\textbf{blue}} and \textcolor{teal}{\textbf{green}} respectively}
\begin{tabular}{c|c|c|ccc}
\hline
\textbf{Method}             & \textbf{Uncertainty Metric} & \textbf{$F_1$}                    & \textbf{$F_{1_{low}}$} & \textbf{$F_{1_{high}}$} & \textbf{$F_{1_{audit}}$} \\ \hline
\multirow{4}{*}{MCD} & Predictive Entropy          & \multirow{4}{*}{77.4}          & 90.2            & 54.4             & 94.3              \\
                            & Predictive Variance         &                                & 83.7            & 70.4             & 92.4              \\
                            & Mutual Information          &                                & 81.8            & 39.9             & 83.8              \\
                            & Expected KL                 &                                & 79.9            & 32.5             & 81.3              \\ \hline
\multirow{4}{*}{Ensemble}   & Predictive Entropy          & \multirow{4}{*}{\textcolor{red}{\textbf{81.0}}} & \textcolor{red}{\textbf{93.8}}   & \textcolor{red}{\textbf{63.0}}    & \textcolor{red}{\textbf{96.7}}     \\
                            & Predictive Variance         &                                & \textcolor{teal}{\textbf{92.6}}            & \textcolor{teal}{\textbf{71.0}}             & \textcolor{teal}{\textbf{96.9}}              \\
                            & Mutual Information          &                                & 92.3            & 59.7             & 95.4              \\
                            & Expected KL                 &                                & 91.7            & 56.6             & 94.8              \\ \hline
\multirow{2}{*}{Confidence} & Entropy (Worst)             & 72.9                           & 87.0            & 58.7             & 94.3              \\
                            & Entropy (Best)              & \textcolor{blue}{\textbf{78.9}}                           & \textcolor{blue}{\textbf{92.8}}            & \textcolor{blue}{\textbf{55.6}}             & \textcolor{blue}{\textbf{96.1}}              \\ \hline
\end{tabular}
\label{tab:MT_results_t0}
\end{table*}
% which would represent a correction of 91.3\% of the network's errors on the test set
 Table \ref{tab:PA_results_t1} presents results when training and testing on the same date pair, which is a common procedure in recent works \cite{ortega2021comparison}. Among the multiple-outcome approaches, the best performing method was Ensemble with an $F_1$ score of 85.8. The reason for its improvement over MCD might be related to using the entire network at each inference pass, without using dropout at inference. As a baseline comparison, we present results for using entropy from a single inference run. Since we repeated this experiment 10 times, we show the worst and best performing cases. Results presented a high variance. Although the best case scenario produced results slightly better compared to Ensemble in terms of $F_{1_{low}}$, results from single inference run, the high variance in results from a single inference run make it unreliable compared to the more robust MCD and Ensemble methods.

We also present results for multiple uncertainty metrics in MCD and Ensemble approaches. In both cases, the best performing metric was predictive entropy, with increases in $F_{1_{low}}$ of up to 8.1\% compared to the other metrics. Predictive variance was the second best metric in both MCD and Ensemble. 


In the case of the best performing multiple-output method (Ensemble) and the best metric (Predictive entropy), we obtained an average $F_1$ score of 85.8. After applying the proposed uncertainty-based methodology, for an Audit Area (AA) of 3\%, we obtained an increased $F_1$ of 96.9 for the samples with low uncertainty, while $F_{1_{high}}$ was much lower (62.3), which indicates that our methodology succeeded in separating the samples that we can trust from the samples that we don't know if they are correct. If we audited the samples with high uncertainty, we would obtain a $F_{1_{audit}}$ of 97.9, with a significantly smaller auditing effort (3\% of the image) compared to the current auditing procedure, where 100\% of the image needs to be audited. Figure \ref{fig:PA_audit_results_t1} presents results for different uncertainty threshold values in the PA site when training and testing in the same date pair, corresponding to a range of AA from 0\% to about 10\%. Results correspond to the best performing multiple-output method and the corresponding best uncertainty metric. $F_1$ before applying the uncertainty methodology is presented in yellow for comparison. $F_{1_{low}}$ increased when increasing AA until a peak value for an AA of approximately 5\%, where its value started to decline. This suggests that the uncertainty threshold needs to be adequately selected to obtain the desired outcomes in the proposed methodology. In contrast, $F_{1_{audit}}$ increased monotonically when increasing AA. Both $F_{1_{low}}$ and $F_{1_{audit}}$ got close to 100\% even for small values of AA. 

Figure \ref{fig:PA_audit_results_t0} presents results for different uncertainty thresholds in the PA site in the more operational case of training with a past date and testing on an upcoming date. In this case, $F_{1_{low}}$ and $F_{1_{audit}}$ increased monotonically when increasing AA, as expected. Compared to $F_1$ before applying the uncertainty methodology, $F_{1_{low}}$ and $F_{1_{audit}}$ were significantly higher even for small values of AA. 

\iffalse
\begin{table*}[ht!]
\caption{Uncertainty results for PA site. $AA=3\%$} %  using past dates and inferring un unseen date
\centering
\begin{tabular}{c|c|c|ccc}
\hline
\textbf{Train date}             & \textbf{Test date} & \textbf{$F_1$} & \textbf{$F_{1_{low}}$} & \textbf{$F_{1_{high}}$} & \textbf{$F_{1_{audit}}$} \\ \hline
$[T_{-1}, T_0]$                       & $[T_{-1}, T_0]$          & 84.1        & 95.3                 & 61.0                  & 96.9                                         \\
$[T_{-2}, T_{-1}]$                       & $[T_{-1}, T_0]$          & 78.3        & 91.5                 & 60.8                  & 95.7                                         \\ \hline
\end{tabular}
\label{tab:PA_uncertainty}
\end{table*}
\fi

\begin{figure*}[ht!]
\centering
\includegraphics[scale=0.45]{figures/4-Experimental/PA/recall_precision_f1_AA_current_date.png}
\caption{Classification metrics for multiple uncertainty threshold values in PA site. Training and testing in [2018, 2019]. Uncertainty method: Ensemble. Uncertainty metric: Predictive Entropy. A sample AA threshold of 3\% is highlighted in \textcolor[HTML]{333333}{gray}.}
\label{fig:PA_audit_results_t1}
\end{figure*}


\begin{figure*}[ht!]
\centering
\includegraphics[scale=0.45]{figures/4-Experimental/PA/recall_precision_f1_AA.png}
\caption{Classification metrics for multiple uncertainty threshold values in PA site. Training in [2017, 2018] and testing in [2018, 2019]. Uncertainty method: Ensemble. Uncertainty metric: Predictive Entropy. A sample AA threshold of 3\% is highlighted in \textcolor[HTML]{333333}{gray}.}
\label{fig:PA_audit_results_t0}
\end{figure*}

\begin{figure*}[ht!]
  \centering
  \begin{tabular}{@{}c@{}}
    \includegraphics[width=1.\linewidth]{figures/4-Experimental/PA/PAMultipleDatesPredictSampleUncertainty2.png} \\[\abovecaptionskip]
    % \small (a) Error areas presented high uncertainty
  \end{tabular}

  % \vspace{0.5\floatsep}
  \vspace*{-0.5\baselineskip}
  
  \begin{tabular}{@{}c@{}}
    \includegraphics[width=1.\linewidth]{figures/4-Experimental/PA/PAMultipleDatesPredictSampleUncertainty1.png} \\[\abovecaptionskip]
    % \small (b) Correct areas presented low uncertainty
  \end{tabular}
  \vspace*{-0.5\baselineskip}
  
  \begin{tabular}{@{}c@{}}
    % \vspace{-3.5\floatsep}
    \includegraphics[width=1.\linewidth]{figures/4-Experimental/PA/colormap.png} \\[\abovecaptionskip]
    % \small (b) Correct areas presented low uncertainty
  \end{tabular}

s  \caption{Qualitative results for PA site. Each row represents a snippet from the test areas. Training in [2017, 2018] and testing in [2018, 2019].}\label{fig:PAqualitative}
\end{figure*}

Figure \ref{fig:PAqualitative} presents qualitative results for the operational scenario where we trained on a past date pair and tested on a new upcoming date pair. In each row we have the optical $T_0$ image, where deforestation has already occurred, followed by the predict probability, the prediction result (White and black for non-deforestation and deforestation, blue and orange for false positive and true positive errors), and the uncertainty estimation map. We can observe that error areas presented high uncertainty values, while correctly classified areas presented low uncertainty values, with the exception of borders surrounding the correctly classified deforestation polygons, which also presented a high uncertainty value. Such borders were expected to present high uncertainty, and the result is consistent with the PRODES protocol, which ignores pixels within a 2-pixel-wide range at the inner and outer boundaries of deforestation polygons. In the second row we see a fail case for the proposed methodology, with a small false negative region presenting low uncertainty values.

The results in Table \ref{tab:PA_results_t0} correspond to training with date pairs from the past and inferring on a new date unseen during training, representing a more realistic operational setting. In this case, the best multiple-output uncertainty method was also Ensemble and the best performing uncertainty metric was also the predictive entropy.

In terms of $F_1$, results were comparable to the upper bound case of training and testing in the same date, with an $F_1$ of 81.4. For an Audit Area (AA) of 3.0\%, $F_{1_{low}}$ improved to 94.4, while $F_{1_{high}}$ was 63.9, which indicates that the proposed uncertainty methodology was also capable of discerning between samples we can trust and samples we don't know if they are correct in the case of training with past date pairs and inferring on a new upcoming date. If we audited the high uncertainty samples, we could get an $F_{1_{audit}}$ of 97.2, which is close to what we obtained in the ideal case of training and testing on the same date, with a slight difference of 0.7\%. 

In this case, entropy results from a single inference run were similar to the Ensemble approach, with the best run being 0.1\% lower in terms of $F_{1_{audit}}$, which indicated that the robustness from Ensemble was more relevant in the more challenging scenario when training and testing in different dates.


\iffalse
\begin{table*}[ht!]
\caption{Uncertainty results for MT site. $AA=3\%$}
\centering
\begin{tabular}{c|c|c|ccc}
\hline
\textbf{Train date}             & \textbf{Test date} & $F_1$ & \textbf{$F_{1_{low}}$} & \textbf{$F_{1_{high}}$} & \textbf{$F_{1_{audit}}$} \\ \hline
$[T_{-1}, T_0]$                       & $[T_{-1}, T_0]$          & 79.2        & 91.7                 & 57.7                  & 94.8                                       \\
$[T_{-2}, T_{-1}]$                       & $[T_{-1}, T_0]$          & 77.8        & 90.4                 & 54.1                  & 94.4                                         \\ \hline
\end{tabular}
\label{tab:MT_uncertainty}
\end{table*}
\fi

\begin{figure*}[ht!]
\centering
\includegraphics[scale=0.45]{figures/4-Experimental/MT/recall_precision_f1_AA_current_date.png}
\caption{Classification metrics for multiple uncertainty threshold values in MT site. Training and testing in [2019, 2020]. Uncertainty method: Ensemble. Uncertainty metric: Predictive Variance. A sample AA threshold of 3\% is highlighted in \textcolor[HTML]{333333}{gray}.}
\label{fig:MT_audit_results_t1}
\end{figure*}


\begin{figure*}[ht!]
\centering
\includegraphics[scale=0.45]{figures/4-Experimental/MT/recall_precision_f1_AA.png}
\caption{Classification metrics for multiple uncertainty threshold values in MT site. Training in [2017, 2018] and testing in [2018, 2019]. Uncertainty method: Ensemble. Uncertainty metric: Predictive Entropy. A sample AA threshold of 3\% is highlighted in \textcolor[HTML]{333333}{gray}.}
\label{fig:MT_audit_results_t0}
\end{figure*}

\begin{figure*}[ht!]
  \centering
  \begin{tabular}{@{}c@{}}
    \includegraphics[width=1.\linewidth]{figures/4-Experimental/MT/MTMultipleDatesPredictSampleUncertainty2.png} \\[\abovecaptionskip]
    % \small (a) Error areas presented high uncertainty
  \end{tabular}

  % \vspace{0.5\floatsep}
  \vspace*{-0.5\baselineskip}
  \begin{tabular}{@{}c@{}}
    \includegraphics[width=1.\linewidth]{figures/4-Experimental/MT/MTMultipleDatesPredictSampleUncertainty1.png} \\[\abovecaptionskip]
    % \small (b) Correct areas presented low uncertainty
  \end{tabular}
  \vspace*{-0.5\baselineskip}
  
  \begin{tabular}{@{}c@{}}
    % \vspace{-3.5\floatsep}
    \includegraphics[width=1.\linewidth]{figures/4-Experimental/PA/colormap.png} \\[\abovecaptionskip]
    % \small (b) Correct areas presented low uncertainty
  \end{tabular}
  \caption{Qualitative results for MT site. Each row represents a snippet from the test areas. Training in [2018, 2019] and testing in [2019, 2020].}\label{fig:MTqualitative}
\end{figure*}


\subsection{Uncertainty Estimation Results in MT Site}\label{sec:results_uncertainty_MT}

Table \ref{tab:MT_results_t1} presents classification results for MT site in the traditional scenario, where the network is trained and tested on the same date. As in the PA site, the best multiple-output uncertainty method was Ensemble, with an increase of 3.6\% $F_1$ compared to MCD. In this case, the best uncertainty metric was the predictive variance in terms of $F_{1_{low}}$ and $F_{1_{audit}}$.  For the Ensemble method and the predictive variance metric, applying the proposed uncertainty methodology, for an AA of 3\%, the $F_{1_{low}}$ improved to 91.7, while $F_{1_{high}}$ produced a significantly lower value, indicating that the methodology was able to separate between predictions we can trust and predictions that the network doesn't know, similarly to PA. If we audited the samples with high uncertainty, we could get a $F_{1_{audit}}$ of 94.8 with minimal auditing effort. Consistently with the PA site, entropy from a single inference run produced results with high variance, with the worst case scenario being 7\% lower compared to MCD and the best outcome 0.2\% higher compared to Ensemble in terms of $F_{1_{low}}$. Such result suggests that using a single inference run may be less reliable and robust compared to the MCD and Ensemble approaches.


Table \ref{tab:MT_results_t0} presents results in a more challenging scenario, which is closer to a real operational setting, for the MT site. As in PA, we trained on an image pair from past dates $[T_{-1}, T_0] = [2018, 2019]$ and tested on a new upcoming date $[T_{-1}, T_0] = [2019, 2020]$.  Results were similar to previous experiments, with Ensemble producing the best results compared to MCD. The best performing uncertainty metric was the predictive entropy. In this operational setting, with the best performing approach we obtained an $F_1$ of 81.0, which represented a minor drop of 0.4\% compared to the ideal case of training and testing on the same date pair. 

Using our proposed methodology to aid the auditing process, for an AA of 3\%, we obtained $F_{1_{low}}$ of 93.8, with a much lower $F_{1_{high}}$, which reinforces our initial hypothesis that uncertainty estimation can help us separate the samples we can trust from the samples the network doesn't know if they are correct. After auditing the high uncertainty samples, we could get an $F_{1_{audit}}$ of 96.7, which is close to the ideal case where we trained and tested on the same date pair. These results were consistent with the ones from PA site.

In the more operational setting of training and testing with different dates, results for entropy from a single inference run were worse compared to MCD and Ensemble methods, even for the best performing case. This reinforces the hypothesis that the robustness from multiple-inference approaches was more critical in the operational case.

In Figures \ref{fig:MT_audit_results_t1} and \ref{fig:MT_audit_results_t0} we present the classification metrics for multiple uncertainty threshold values when training and testing in the same date and when training and testing in different dates, respectively. These results correspond to the best performing approach in each case. We present $F_1$ before applying the uncertainty methodology in yellow for comparison. We observe a similar behavior to PA site, with $F_{1_{low}}$ and $F_{1_{audit}}$ increasing when increasing AA. The auditor might select a low AA such as 3\% and still get high $F_1$ values, or use a more conservative approach and select a higher AA with correspondingly higher $F_1$ metrics.

Figure \ref{fig:MTqualitative} presents qualitative results for the MT site in the scenario where we trained on a past date pair [2018, 2019] and tested on an upcoming date pair [2019, 2020]. In each row, a snippet from the test area is presented. In the classification predictions, white and black represent correctly classified non-deforestation and deforestation classes, while blue and orange represent false positive and true positive errors. The first snippet corresponds to a region with a significant amount of false negative errors. The uncertainty map produced high values in the error areas, while it produced lower values in the correctly classified regions, indicating its potential to detect areas which need to be revised by an auditing expert. In the second snippet, a region with almost no classification errors produced generally low uncertainty values, with the exception of border areas surrounding deforestation polygons as in PA, as expected.



\subsection{Polygon Analysis}\label{sec:results_polygon}

So far, we presented pixel-level results. However, auditing experts usually work with polygons instead of individual pixels. In this section, we analyze the produced uncertainty map in terms of uncertainty polygon count, which represents the auditing effort. Uncertainty polygons were obtained by applying a specific threshold to the uncertainty metric value for each pixel location. In this analysis, we applied an uncertainty threshold corresponding to $AA=3\%$. The total count of uncertainty polygons for PA site was 13713. To reduce the effort in the auditing process, we hypothesize that the auditor could start considering the largest uncertainty polygons, and account for most of the auditing process without the need to consider the smallest polygons, which correspond to the majority of uncertainty polygons. 

In Figure \ref{fig:PA_polygon_analysis}, we present results for polygon analysis in PA. In the horizontal axis, we grouped the uncertainty polygons by their individual polygon size. In Figure \ref{fig:PA_polygon_analysis}(a), we present the number of polygons for different values of polygon size. We observe that the vast majority of polygons (92.3\%) have an individual polygon size in the smallest horizontal bin, which corresponds to polygons with an individual size lower than 625 pixels. Such small polygons might not be of use for auditors, considering that the annotation protocol from PRODES ignores polygons with an area lower than $6.25ha$, corresponding to 625 pixels. If we ignore such small polygons, we would be left with 1058 uncertainty polygons, which might be more convenient for the experts to audit. Figure \ref{fig:PA_polygon_analysis}(b) presents in blue the accumulated area for each polygon size bin, and in red the cumulative area counting from the largest polygon size on the right to the smallest on the left. The cumulative area does not start from 0\% because there are some polygons with size larger than the figure limits. If we dropped the uncertainty polygons in the smallest bin, which are too small for auditors to consider, we would reduce the amount of uncertainty polygons to be audited from 13719 to 1058, while retaining 78.1\% of the uncertainty area. 
Figure \ref{fig:MT_polygon_analysis} presents results for polygon analysis in MT site. In this case, the total count of uncertainty polygons was 12320. As in the PA site, if we removed the uncertainty polygons with an area lower than $6.25ha$, we would reduce their count to 703 polygons, while retaining 82.7\% of the uncertainty area. 

% from the largest polygon size on the right to the smallest on the left

% \end{figure*}
\begin{figure*}
  \centering
  \begin{tabular}{@{}c@{}}
    \includegraphics[scale=0.4]{figures/4-Experimental/PA/polygon_analysis_percentage_split1.png} \\[\abovecaptionskip]
    \small (a) Number of polygons for different values of individual polygon size
  \end{tabular}

  \vspace{\floatsep}

  \begin{tabular}{@{}c@{}}
    \hspace*{28pt}\includegraphics[scale=0.4]{figures/4-Experimental/PA/polygon_analysis_percentage_split2.png} \\[\abovecaptionskip]
    \small (b) Bin area percentage in blue, and cumulative area percentage in red, \\ for different values of individual polygon size
  \end{tabular}

  \caption{Polygon analysis for PA site.}\label{fig:PA_polygon_analysis}
\end{figure*}

\begin{figure*}
  \centering
  \begin{tabular}{@{}c@{}}
    \includegraphics[scale=0.4]{figures/4-Experimental/MT/polygon_analysis_percentage_split1.png} \\[\abovecaptionskip]
    \small (a) Number of polygons for different values of individual polygon size
  \end{tabular}

  \vspace{\floatsep}

  \begin{tabular}{@{}c@{}}
    \hspace*{20pt}\includegraphics[scale=0.4]{figures/4-Experimental/MT/polygon_analysis_percentage_split2.png} \\[\abovecaptionskip]
    \small (b) Bin area percentage in blue, and cumulative area percentage in red, \\ for different values of individual polygon size
  \end{tabular}

  \caption{Polygon analysis for MT site.}\label{fig:MT_polygon_analysis}
\end{figure*}

\iffalse
\subsection{Varying Inference Times}\label{sec:results_inference_times}

To assess the influence of varying the amount of inference times, we evaluated the proposed approach for $n=[1, 5, 10, 30, 50]$. Tables \ref{tab:PA_inference_runs} and \ref{tab:MT_inference_runs} show results for PA and MT study areas. In general, results were similar for all values of $n>1$. However, for $n=1$ there was a drop of up to 0.9\% in classification metrics without considering uncertainty ($F_1$). This was consistent with a drop of up to  2.1\% in $F_{1_{low}}$ and 1.2\% in $F_{1_{audit}}$. Thus, the results indicated that $n\geq 5$ produced improvements in the outcomes of the uncertainty methodology. Furthermore, using $n=1$ resulted in larger performance drops for MT site compared to PA site, which indicated that the use of multiple inference runs was more useful for sites with a lower absolute performance.

\begin{table}[ht!]
\caption{Results for varying $N$ in PA site. $AA=3\%$} %  using past dates and inferring un unseen date
\centering
\begin{tabular}{c|c|ccc}
\hline
\textbf{$n$} & \textbf{$F_1$} & \textbf{$F_{1_{low}}$} & \textbf{$F_{1_{high}}$} & \textbf{$F_{1_{audit}}$} \\ \hline
1          & 84.3        & 94.5                 & 63.1                  & 96.4                     \\
5          & 84.5        & 95.3                 & 62.3                  & 96.9                     \\
10          & 84.5        & 95.4                 & 61.7                  & 97.0                     \\
30          & 84.3        & 95.5                 & 60.9                  & 97.0                     \\
50          & 84.3        & 95.5                 & 61.0                  & 97.0                     \\ \hline

\end{tabular}
\label{tab:PA_inference_runs}
\end{table}


\begin{table}[ht!]
\caption{Results for varying $N$ in MT site. $AA=3\%$} %  using past dates and inferring un unseen date
\centering
\begin{tabular}{c|c|ccc}
\hline
\textbf{$n$} & \textbf{$F_1$} & \textbf{$F_{1_{low}}$} & \textbf{$F_{1_{high}}$} & \textbf{$F_{1_{audit}}$} \\ \hline
1          & 78.9        & 89.5                 & 60.7                  & 93.5                     \\
5          & 79.8        & 91.6                 & 58.8                  & 94.7                     \\
10          & 79.8        & 91.4                 & 58.8                  & 94.6                     \\
30          & 79.6        & 91.5                 & 58.1                  & 94.7                     \\
50          & 79.6        & 91.5                 & 58.1                  & 94.7                     \\ \hline

\end{tabular}
\label{tab:MT_inference_runs}
\end{table}
\fi
\iffalse
\subsection{Comparison of Uncertainty Metrics}\label{sec:results_metrics}

We compared the proposed uncertainty methodology for multiple uncertainty metrics. Table \ref{tab:PA_uncertainty_metrics} presents results for PA site. In this case, the best metric was predictive entropy, with increases in $F_{1_{low}}$ of up to 8.1\% compared to the other metrics. Predictive variance was the second best metric. Table \ref{tab:MT_uncertainty_metrics} presents results for MT site. In this case, the best metrics were predictive variance and predictive entropy, with the former having slightly better results in $F_{1_{low}}$ and the latter being better in $F_{1_{audit}}$. Overall, the most consistent metric across study areas was predictive entropy.



\begin{table}[ht!]
\caption{Results for multiple uncertainty metrics in PA site. $AA=3\%$} 
\centering
\begin{tabular}{c|c|ccc}
\hline
\textbf{Metric}     & \textbf{$F_1$} & \textbf{$F_{1_{low}}$} & \textbf{$F_{1_{high}}$} & \textbf{$F_{1_{audit}}$} \\ \hline
Predictive Entropy  & \multirow{3}{*}{84.1} & 95.3                 & 61.0                  & 96.9                     \\
Predictive Variance &                       & 90.9                 & 77.5                  & 96.2                   \\
Mutual Information  &                       & 88.2                 & 34.0                  & 89.3                   \\
Expected KL  &                       & 87.2                 & 26.7                  & 88.0                   \\ \hline
\end{tabular}
\label{tab:PA_uncertainty_metrics}
\end{table}

\begin{table}[ht!]
\caption{Results for multiple uncertainty metrics in MT site. $AA=3\%$} 
\centering
\begin{tabular}{c|c|ccc}
\hline
\textbf{Metric}     & \textbf{$F_1$} & \textbf{$F_{1_{low}}$} & \textbf{$F_{1_{high}}$} & \textbf{$F_{1_{audit}}$} \\ \hline
Predictive Entropy  & \multirow{3}{*}{79.2} & 91.7                 & 57.7                  & 94.8                     \\
Predictive Variance &                       & 92.2                 & 64.0                  & 95.9                   \\
Mutual Information  &                       & 89.3                 & 43.3                  & 92.0                   \\
Expected KL  &                       & 87.8                 & 39.4                  & 90.4                   \\ \hline
\end{tabular}
\label{tab:MT_uncertainty_metrics}
\end{table}
\fi

\iffalse
\textcolor{red}{\subsection{Comparison of Uncertainty Estimation Methods}\label{sec:results_methods}}

\textcolor{red}{The main advantage of MCD is that it only requires training the network one time, due to its ability to produce different outcomes at inference using the same network. In this section, we compared MCD with an ensemble-based approach as an alternative uncertainty estimation method, where we trained a number $n$ of networks with random weight initialization and computed the predictive entropy as uncertainty metric over the inference runs from those networks.} 

\textcolor{red}{Tables \ref{tab:PA_comparison} and \ref{tab:MT_comparison} present results for both uncertainty methods in PA and MT sites. In general, the ensemble-based approach produced improvements in all cases, with increases in $F_{1_{low}}$ and $F_{1_{audit}}$ of up to 3.4\% and 1.5\% respectively. The highest improvements were obtained when training with a past date pair $[T_{-2}, T_{-1}]$ and testing on the current date pair $[T_{-1}, T_0]$. Although the ensemble approach produced significant gains, its computational cost is $n$ times larger during training compared to MCD.}



\begin{table*}[ht!]
\caption{Comparison of Uncertainty Methods for PA site. $AA=3\%$}
\centering
\begin{tabular}{c|cc|c|ccc}
\hline
\textbf{Method}             & \textbf{Train  date} & \textbf{Test date} & $F_1$ & \textbf{$F_{1_{low}}$} & \textbf{$F_{1_{high}}$} & \textbf{$F_{1_{audit}}$} \\ \hline
\multirow{2}{*}{MCD} & $[T_{-1}, T_0]$                   & $[T_{-1}, T_0]$                 & 84.1        & 95.3                 & 61.0                  & 96.9                     \\
                            & $[T_{-2}, T_{-1}]$                   & $[T_{-1}, T_0]$                 & 78.3        & 91.5                 & 60.8                  & 95.7                   \\ \hline
\multirow{2}{*}{Ensemble}   & $[T_{-1}, T_0]$                   & $[T_{-1}, T_0]$                 & 85.8        & 96.9                 & 62.4                  & 97.9                   \\
                            & $[T_{-2}, T_{-1}]$                   & $[T_{-1}, T_0]$                 & 81.4        & 94.5                 & 64.0                  & 97.2                  
\end{tabular}
\label{tab:PA_comparison}
\end{table*}


\begin{table*}[ht!]
\caption{Comparison of Uncertainty Methods for MT site. $AA=3\%$}
\centering
\begin{tabular}{c|cc|c|ccc}
\hline
\textbf{Method}             & \textbf{Train  date} & \textbf{Test date} & $F_1$ & \textbf{$F_{1_{low}}$} & \textbf{$F_{1_{high}}$} & \textbf{$F_{1_{audit}}$} \\ \hline
\multirow{2}{*}{MCD} & $[T_{-1}, T_0]$                   & $[T_{-1}, T_0]$                 & 79.2        & 91.7                 & 57.7                  & 94.8                     \\
                            & $[T_{-2}, T_{-1}]$                   & $[T_{-1}, T_0]$                 & 77.8        & 90.4                 & 54.1                  & 94.4                   \\ \hline
\multirow{2}{*}{Ensemble}   & $[T_{-1}, T_0]$                   & $[T_{-1}, T_0]$                 & 81.4        & 94.1                 & 63.8                  & 96.6                   \\
                            & $[T_{-2}, T_{-1}]$                   & $[T_{-1}, T_0]$                 & 81.0        & 93.8                 & 63.1                  & 96.7                  
\end{tabular}
\label{tab:MT_comparison}
\end{table*}
\fi



% Error areas presented high uncertainty, and correct areas presented low uncertainty


% \begin{figure}[h!]
% \centering
% \includegraphics[scale=0.4]{figures/4-Experimental/PA/qualitative_PA.png}
% \caption{Residual block with dropout.}
% \label{residual_block}
% \end{figure}

% \begin{table*}[]
% \centering
% \begin{tabular}{c|c|cccc}
% \hline
% \textbf{Train date}             & \textbf{Test date} & \textbf{mAP} & \textbf{$F_1$} & \textbf{Precision} & \textbf{Recall} \\ \hline
% 2018-2019                       & 2018-2019          & 91.2         & 82.7        & 81.3               & 84.1            \\
% 2015-2016, 2016-2017, 2017-2018 & 2018-2019          & 88.6         & 80.5        & 91.1               & 72.0            \\
% 2016-2017, 2017-2018            & 2018-2019          & 86.7         & 78.3        & 87.9               & 70.5            \\
% 2017-2018                       & 2018-2019          & 85.7         & 77.2        & 82.6               & 72.5            \\ \hline
% \end{tabular}
% \end{table*}





% \begin{figure}[h!]
% \centering
% \includegraphics[scale=0.4]{figures/4-Experimental/MT/MTResUnet uncertainty predictive entropy.png}
% \caption{Residual block with dropout.}
% \label{residual_block}
% \end{figure}

%%\subsection{Polygon Analysis}

%%So far, we presented pixel-level results. However, the auditing experts usually work with polygons instead of individual pixels. In this section, we analyze the produced uncertainty map in terms of uncertainty polygon count, which represents the auditing effort. The total count of uncertainty polygons for PA site was 17296. In Figure \ref{fig:PA_polygon_analysis}, we present results for polygon analysis in PA. In yellow, we present the number of polygons for multiple individual polygon size in pixels. We observe that the vast majority of polygons (94\%) have an individual polygon size smaller than 625 pixels. Such small polygons might not be of use for auditors, considering that the annotation protocol from PRODES ignores polygons with an area lower than $6.25ha$, corresponding to 625 pixels. If we dropped such small polygons, we would be left with 1045 uncertainty polygons, which might be more convenient for the experts to audit. The figure presents in blue the accumulated area for each polygon size bin, and in red the total cumulative area counting from the largest polygon size on the right to the smallest on the left. If we dropped the uncertainty polygons in the smallest bin, which are too small for auditors to consider, we would reduce the amount of uncertainty polygons to be audited from 17296 to 1045, while retaining around 80\% of the uncertainty area. 
%%Figure \ref{fig:MT_polygon_analysis} presents results for polygon analysis in MT site. In this case, the total count of uncertainty polygons was 26882. As in the PA site, if we removed the uncertainty polygons with an area lower than $6.25ha$, we would reduce their count to 833, while retaining around 80\% of the uncertainty area. 

%%\begin{figure*}[ht!]
%%    \centering
%%		\includegraphics[scale=0.40]{figures/4-Experimental/PA/polygon_analysis_percentage.png}
%%	\caption{Polygon analysis for PA site.}
%%\label{fig:PA_polygon_analysis} % to-do: improve blocks?
%%\end{figure*}

%%\begin{figure*}[ht!]
%%    \centering
%%		\includegraphics[scale=0.40]{figures/4-Experimental/MT/polygon_analysis_percentage.png}
%%	\caption{Polygon analysis for MT site.}
%%\label{fig:MT_polygon_analysis} % to-do: improve blocks?
%%\end{figure*}


   \chapter{Summary and Next Steps}\label{sect:Conclusions}

In this work, preliminary results were presented for uncertainty estimation for deforestation detection in the Brazilian amazon using optical satellite images. A comparison of multiple uncertainty estimation methods including confidence-based approaches, MCD and ensembles was presented. 

As next steps, more uncertainty estimation methods will be assessed including additional confidence-based approaches and evidential deep learning. Furthermore, a novel uncertainty estimation approach for transformer base networks will be presented. Finally, experiments will be extended to an additional application, namely secondary vegetation mapping. % implementar no brazil data cube
% colocar cada uma das tarefas
% lista as tarefas
% coloca uma tabela cronograma

Specifically, the following list of tasks will be performed as next steps:

\begin{enumerate}
\item Implement evidential deep learning for deforestation mapping using ResUnet CNN as a base network
\item Implement transformer as a base network using the confidence-based, MCD and ensemble methods
\item Implement evidential deep learning for transformer as a base network
\item Repeat experiments for secondary vegetation mapping
\item Implement the proposed methods in a real operational environment using Brazilian Data Cube and SITS
\item Write a research paper
\item Final document writing
\end{enumerate}

The timetable for the remaining time of the doctoral research project is presented in Figure \ref{fig:timetable}.

During this doctorate, a total of 7 research works have been published, including 3 papers in research journals and 4 papers in international conferences \cite{martinez2021fully, martinez2021open, chamorro2021towards, martinez2022comparison, rogozinski20223d, rogozinski2021exploring, sanches2020first}. An additional work has already been accepted for publication in a research journal \cite{10042432}, and another work containing the preliminary results presented in this document is in the submission process for a research journal \cite{chamorro2023semi}.

\begin{figure*}[h!]
\centering
% \includegraphics[scale=0.6]{figures/method/model.png}
\includegraphics[scale=0.65]{figures/5-NextSteps/cronograma.PNG}
\caption{Timetable for the remaining time of the doctoral study.}
\label{fig:timetable}
\end{figure*}
%% lista de tarefas
% evidential learning para cnn
% transformer (elegir 1) e os mesmos metodos aplicados para a cnn (entropia, mcd, ensemble)
% evidential com o transformer
% rodar tudo isso para vegetacao secundaria (nao e estritamente necessario) (desejavel)
% implementacao no brazil data cube/sits (desejavel) 
% escrever um paper (resumo da tese)
% no fim, colocar uma lista das publicacoes e o ultimo, submitted

%% cronograma desejado




% The proposed uncertainty estimation method was able to separate between inferred samples that can be trusted (samples with low uncertainty) and samples that the network does not know if they are correct (samples with high uncertainty), which could be sent to an expert annotator for manual auditing. For a low percentage of samples to be audited of 3\%, the remaining 97\% of low uncertainty samples produced improvements from 86.5\% to 96.2\% average F1 score for the PA site, and from 81.8\% to 91.0\% for the MT site. Furthermore, if the aforementioned 3\% of high uncertainty samples were perfectly audited, the improvement would be of up to 97.8\% F1 score, which is higher than what is required by INPE for an operational implementation of automatic deforestation mapping.

% Furthermore, our experiments indicated that we can train the network on past date pairs, and infer on a new upcoming date without the need to use ground truth from the upcoming date during training. Remarkably, training with past dates produced only a slight decrease in F1 score from 97.8 to 97.6 in PA site, and an increase in F1 score from 93.8 to 95.8 in MT site. 


% \subsection*{Future directions}

% As future directions, we will investigate recent base architectures such as vision transformer \cite{dosovitskiy2020image, ranftl2021vision}. Besides, we will consider other uncertainty estimation methods such as evidential deep learning \cite{sensoy2018evidential}.


   \arial
   \bibliography{references}
   \normalfont
%   \input{chapter-7}
\end{document}