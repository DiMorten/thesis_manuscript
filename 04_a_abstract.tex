\abstract{
%Atualmente, o monitoramento oficial do desmatamento na Amazônia brasileira é feito por especialistas humanos que avaliam visualmente imagens de sensoriamento remoto e rotulam cada pixel individual como desmatamento ou não desmatamento. Essa metodologia é obviamente cara e demorada. A razão para não usar métodos totalmente automáticos para a tarefa é a necessidade da maior precisão possível nos números oficiais de desmatamento. Neste trabalho, propomos uma alternativa semiautomática baseada em aprendizado profundo, na qual primeiro treinamos uma rede totalmente convolucional com imagens e referências existentes de anos anteriores e empregamos essa rede para realizar a detecção de desmatamento em imagens recentes. Após a inferência, estimamos a incerteza nos resultados em nível de pixel da rede e assumimos que os resultados da classificação de baixa incerteza podem ser confiáveis. As demais regiões de alta incerteza, que correspondem a uma pequena porcentagem da área de teste, são então submetidas a um procedimento de auditoria, realizado visualmente por um especialista humano. Desta forma, o esforço de rotulagem manual é bastante reduzido. Também avaliamos a metodologia de mapeamento da vegetação secundária nos biomas Amazônia e Cerrado. Diferentes métodos de incerteza foram comparados, incluindo abordagens baseadas em confiança, Monte Carlo Dropout (MCD), ensembles e deep learning evidencial. Redes convolucionais e baseadas em transformadores foram usadas como modelos de base. Resultados preliminares são apresentados para detecção de desmatamento. Supondo que a avaliação visual esteja correta, as acurácias obtidas com tal metodologia nas áreas de teste são muito altas -- de até 96,9\% média $F_1$-score, para uma área a ser auditada de apenas 3\%. Avaliamos a metodologia proposta em duas áreas de estudo desafiadoras na Amazônia brasileira. O código está disponível em \url{https://github.com/DiMorten/deforestation_uncertainty}.

Atualmente, o monitoramento oficial do desmatamento na Amazônia brasileira é feito por especialistas humanos que avaliam visualmente imagens de sensoriamento remoto e rotulam cada pixel individual como desmatamento ou não desmatamento. Essa metodologia é obviamente cara e demorada. A razão para não usar métodos totalmente automáticos para a tarefa é a necessidade da maior precisão possível nos números oficiais de desmatamento. Neste trabalho, é proposta uma alternativa semiautomática baseada em aprendizado profundo, na qual uma rede neural profunda é primeiro treinada com imagens e referências existentes de anos anteriores e empregada para realizar a detecção de desmatamento em imagens recentes. Após a inferência, a incerteza nas predições da rede é estimada e assume-se que os resultados com baixa incerteza são confiáveis. As regiões restantes de alta incerteza, que correspondem a uma pequena porcentagem da área de teste, são então submetidas à pós-classificação, por exemplo, em um procedimento de auditoria realizado visualmente por um especialista humano. Desta forma, o esforço de rotulagem manual é bastante reduzido. A avaliação dessa metodologia para o mapeamento da vegetação secundária nos biomas Amazônia e Cerrado também está prevista. Diferentes métodos de incerteza serão comparados, incluindo abordagens baseadas em confiança, Monte Carlo Dropout (MCD), \textit{ensembles} e \textit{evidencial learning}. Redes totalmente convolucionais e \textit{transformers} serão usadas como arquiteturas básicas. Neste documento são apresentados resultados preliminares para detecção de desmatamento. Supondo que a avaliação visual esteja correta, as acurácias obtidas até agora com essa metodologia para detecção de desmatamento em duas áreas de estudo desafiadoras na Amazônia brasileira são muito altas -- de até 96,9\% média $F_1$-score, para uma área a auditar de apenas 3\%. O código atual está disponível em \url{https://github.com/DiMorten/deforestation_uncertainty}.
}